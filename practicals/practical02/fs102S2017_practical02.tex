% CS 111 style
% Typical usage (all UPPERCASE items are optional):
%       \input 111pre
%       \begin{document}
%       \MYTITLE{Title of document, e.g., Lab 1\\Due ...}
%       \MYHEADERS{short title}{other running head, e.g., due date}
%       \PURPOSE{Description of purpose}
%       \SUMMARY{Very short overview of assignment}
%       \DETAILS{Detailed description}
%         \SUBHEAD{if needed} ...
%         \SUBHEAD{if needed} ...
%          ...
%       \HANDIN{What to hand in and how}
%       \begin{checklist}
%       \item ...
%       \end{checklist}
% There is no need to include a "\documentstyle."
% However, there should be an "\end{document}."
%
%===========================================================
\documentclass[11pt,twoside,titlepage]{article}
%%NEED TO ADD epsf!!
\usepackage{threeparttop}
\usepackage{graphicx}
\usepackage{latexsym}
\usepackage{color}
\usepackage{listings}
\usepackage{fancyvrb}
%\usepackage{pgf,pgfarrows,pgfnodes,pgfautomata,pgfheaps,pgfshade}
\usepackage{tikz}
\usepackage[normalem]{ulem}
\tikzset{
    %Define standard arrow tip
%    >=stealth',
    %Define style for boxes
    oval/.style={
           rectangle,
           rounded corners,
           draw=black, very thick,
           text width=6.5em,
           minimum height=2em,
           text centered},
    % Define arrow style
    arr/.style={
           ->,
           thick,
           shorten <=2pt,
           shorten >=2pt,}
}
\usepackage[noend]{algorithmic}
\usepackage[noend]{algorithm}
\newcommand{\bfor}{{\bf for\ }}
\newcommand{\bthen}{{\bf then\ }}
\newcommand{\bwhile}{{\bf while\ }}
\newcommand{\btrue}{{\bf true\ }}
\newcommand{\bfalse}{{\bf false\ }}
\newcommand{\bto}{{\bf to\ }}
\newcommand{\bdo}{{\bf do\ }}
\newcommand{\bif}{{\bf if\ }}
\newcommand{\belse}{{\bf else\ }}
\newcommand{\band}{{\bf and\ }}
\newcommand{\breturn}{{\bf return\ }}
\newcommand{\mod}{{\rm mod}}
\renewcommand{\algorithmiccomment}[1]{$\rhd$ #1}
\newenvironment{checklist}{\par\noindent\hspace{-.25in}{\bf Checklist:}\renewcommand{\labelitemi}{$\Box$}%
\begin{itemize}}{\end{itemize}}
\pagestyle{threepartheadings}
\usepackage{url}
\usepackage{wrapfig}
% \usepackage{hyperref}
\usepackage[hidelinks]{hyperref}
%=========================
% One-inch margins everywhere
%=========================
\setlength{\topmargin}{0in}
\setlength{\textheight}{8.5in}
\setlength{\oddsidemargin}{0in}
\setlength{\evensidemargin}{0in}
\setlength{\textwidth}{6.5in}
%===============================
%===============================
% Macro for document title:
%===============================
\newcommand{\MYTITLE}[1]%
   {\begin{center}
     \begin{center}
     \bf
     FS 102 \\Software Everywhere\\
     Spring 2017\\
     \medskip
     \end{center}
     \bf
     #1
     \end{center}
}
%================================
% Macro for headings:
%================================
\newcommand{\MYHEADERS}[2]%
   {\lhead{#1}
    \rhead{#2}
    \immediate\write16{}
    \immediate\write16{DATE OF HANDOUT?}
    \read16 to \dateofhandout
    \lfoot{\sc Handed out on \dateofhandout}
    \immediate\write16{}
    \immediate\write16{HANDOUT NUMBER?}
    \read16 to\handoutnum
    \rfoot{Handout \handoutnum}
   }

%================================
% Macro for bold italic:
%================================
\newcommand{\bit}[1]{{\textit{\textbf{#1}}}}

%=========================
% Non-zero paragraph skips.
%=========================
\setlength{\parskip}{1ex}

%=========================
% Create various environments:
%=========================
\newcommand{\PURPOSE}{\par\noindent\hspace{-.25in}{\bf Purpose:\ }}
\newcommand{\SUMMARY}{\par\noindent\hspace{-.25in}{\bf Summary:\ }}
\newcommand{\DETAILS}{\par\noindent\hspace{-.25in}{\bf Details:\ }}
\newcommand{\HANDIN}{\par\noindent\hspace{-.25in}{\bf Hand in:\ }}
\newcommand{\SUBHEAD}[1]{\bigskip\par\noindent\hspace{-.1in}{\sc #1}\\}
%\newenvironment{CHECKLIST}{\begin{itemize}}{\end{itemize}}


\usepackage[compact]{titlesec}

\begin{document}
\MYTITLE{Practical 2\\Assigned: Wednesday, February 1, 2017\\Due: Wednesday, February 8, 2017 at the start of class\\``Checkmark'' grade}

\vspace*{-.2in}
\section*{Introduction}

Writers and presenters on the cutting-edge of technology often use a version control system to manage most of the
artifacts produced during the phases of drafting and delivering an article or a talk. In this course, we will always use
a GitHub repository to host the web site that will feature our writing and presentations. In this practical assignment,
you will learn how to create a preliminary version of your mobile-ready web site and take the first step towards adding
some of your existing content to that site. After finishing this assignment you should be able to view both a local
version of your web site running on your development computer and a publicly available version of the site that is
hosted by GitHub. As you are completing this practical assignment, please make sure that you consider the following
admonitions about using GitHub to complete a writing assignment.

\vspace*{-.1in}
\begin{itemize}
  \setlength{\itemsep}{-.01in}

\item {\bf If possible, use the laboratory computers.} If it is absolutely necessary for you to work on a different
  machine, be sure to regularly transfer your programs to the Alden machines and check their correctness. Please
  remember that, as stated in the syllabus, students should try to complete assignments using the specialized
  workstations in the laboratory. If you cannot use a laboratory computer, then, when you are asking questions, please
  carefully explain the setup of your laptop to a teaching assistant or to the course instructor.

\item {\bf Follow each step carefully.} Slowly read each sentence in every assignment sheet, making sure that you
  precisely follow each instruction. Take notes about each step that you attempt, recording your questions and ideas
  and the challenges that you faced. If you are stuck, then please tell a teaching assistant or instructor what step
  you recently completed.

\item {\bf Regularly ask and answer questions.} Please log into Slack at the start of a class or practical session and
  then join the appropriate channel. If you have a question about one of the steps in an assignment, then you can post
  it to the designated channel. Or, you can ask a student sitting next to you or talk with a teaching assistant or the
  course instructor.

\item {\bf Store your files in Git}. Starting with this laboratory assignment, you will be responsible for storing all
  of your files in a Git repository. Please verify that you have saved your source code in your Git repository by
  typing ``{\tt git status}'' and ensuring everything is up to date.

\item {\bf Keep all of your files!} Don't delete your programs, output files, and reports after you hand them in---you
  will need them again later when you study for the quizzes and examinations and work on the other laboratory,
  practical, and final project assignments.

\item {\bf Back up your files regularly}. Use a flash drive, Google Drive, or your favorite backup method to keep a
  copy of your files in reserve. In the event of a system failure, you are responsible for ensuring that you have
  access to a recent backup copy of all your files.

\end{itemize}

\section*{Configuring Git and GitHub}

During this practical assignment and subsequent assignments, we will securely communicate use GitHub to host our writing
and presentations. If you did not complete these steps in a previous practical assignment, then you must now ensure that
you have configured your accounts on the departmental servers and the GitHub service. Throughout this assignment, you
should refer to the following web site for more details: \url{https://guides.github.com/activities/hello-world/}. As you
will be required to use Git in the remaining writing, speaking, and practical assignments and during the class sessions,
please be sure to keep a record of all of the steps that you complete and the challenges that you face. If you confront
and then ultimately resolve a confusing issue, please share your experiences in the {\tt \#practicals} channel of Slack
team for this course.

\begin{enumerate}

  \item If you do not already have a GitHub account, then please go to the GitHub web site and create one---make
    sure that you use your {\tt allegheny.edu} email address so that you can join the GitHub Educational Community as
    this step becomes necessary. Also, please make sure that you add a description of yourself and an appropriate
    professional photograph to your GitHub profile. For examples of what a professional GitHub profile might look like,
    please consider studying {\tt https://github.com/una} and {\tt https://github.com/gkapfham}.

  \item If you have never done so before, you must use the {\tt ssh-keygen} program to create secure-shell keys that you
    can use to support your communication with the GitHub servers. But, to start, this task requires you to type
    commands in a program that is known as a terminal. To run it, on the left side of your screen, click on the icon
    that contains the ``{\tt >}'' symbol. Alternatively, you can type the ``Super'' key, start typing the word
    ``terminal'', and then select that program. Another way to open a terminal involves typing the key combination {\tt
    <Ctrl>-<Alt>-t}.

  \item If you have not done so already, you will now need to run the {\tt ssh-keygen} command in your terminal window.
    Follow the prompts to create your keys and save them in the default directory (press ``Enter'' after you are
    prompted: ``{\tt Enter file in which to save the key ...  :}'', then press ``Enter'' twice if you do not wish to
    create a passphrase at this time or type your selected passphrase if you do). What files does {\tt ssh-keygen}
    produce? Where does this program store these files by default? Do you have questions about this step?

  \item Once you have created your ssh keys, you should raise your hand to invite either a teaching assistant or the
    course instructor to help you with the next steps. First, you must log into GitHub and look in the right corner for
    an account avatar with a down arrow. Click on this link and then select the ``Settings'' option. Now, scroll down
    until you found the ``SSH and GPG keys'' label on the left, click create a new ``SSH key'', and then upload your ssh
    key to GitHub. You can copy your to SSH key to the clipboard by going to the terminal and typing ``{\tt cat
    \textasciitilde{}/.ssh/id\_rsa.pub}'' command and then highlighting this output. When you are completing this step
    in your terminal window, please make sure that you only highlight the letters and numbers in your key---if you
    highlight any extra symbols or spaces then this step may not work correctly. Then, paste this into the text field in
    your web browser.

  \item Again, when you are completing these steps, please make sure that you take careful notes about the inputs,
    outputs, and behavior of each command. If there is something that you do not understand, then please ask the course
    instructor or the teaching assistant about it.

  \item Since this is your first practical assignment and you are still learning how to use the appropriate software,
    don't become frustrated if you make a mistake. Instead, use your mistakes as an opportunity for learning both about
    the necessary technology and the background and expertise of the other students in the class, the teaching
    assistants, and the course instructor. Remember, you can use Slack to talk with the instructor by using ``{\tt
    @gkapfham}'' in a channel.

\end{enumerate}

\vspace*{-.15in}
\section*{Creating a Mobile-Ready Web Site}

As you complete the next part of this practical assignment please make sure that you follow each step carefully. If you
make a mistake in one of these steps it may require you to start over and follow all of the steps again. If you are not
sure how to do one of the requested actions, then please ask the course instructor or a teaching assistant. To start
this phase of the assignment, please visit and read the following web site:
\url{https://github.com/daattali/beautiful-jekyll}.

\vspace*{-.15in}
\begin{enumerate}

  \itemsep0in

  \item After reading the aforementioned web site, you will know that ``Beautiful Jekyll'' is a template for creating a
    mobile ready web site by using the Jekyll static site generator and the Bootstrap responsive layout framework.
    Before you start the next steps, please watch the movie on this web site to get a sense as to how you are going to
    download and customize this theme by using GitHub. Next, you should fork the repository by clicking the appropriate
    button that is in the top right corner of your web browser. What do you think that forking accomplishes?

  \item Now that you have forked this repository, you need to go into its settings and rename it to ``{\tt <your GitHub
    user name>.github.io}''. Please see the course instructor or a teaching assistant if you do not know your GitHub
    user name (which should be, if possible, the same as your {\tt allegheny.edu} email user name) or you are not sure
    how to rename the repository.

  \item Using SSH --- and not HTTPS! --- you should now clone the repository to your laboratory computer. You will do
    this by clicking the green ``Clone'' button, copying the address to the clipboard, and then typing ``{\tt git
    clone}'' and pasting the address in your terminal window. At this point, you should see the download of the source
    code for your new web site to your workstation. Ask the instructor for help if you think that this did not work
    correctly.

  \item Next, you should change into the directory for your web site's repository and type the command ``{\tt bundle
    install --path \textasciitilde{}/.gem}'' and wait for all of the needed packages to install correctly on your
    computer workstation. Since this command may take a long time to run, you should start it now and then, if
    necessary, move to the next step in the assignment.

  \item Once the packages have installed correctly, you can type ``{\tt bundle exec jekyll server}'' to transform the
    markdown of your web site into HTML and to start a local web server that you can use to preview your site. If this
    command works, then you can access the web server by typing {\tt http://127.0.0.1:4000} into your web browser. Can
    you see a web site now?

  \item After you have gained access to the local (i.e., ``development'') version of your web site, it is also a good
    idea to view the version that is currently hosted by GitHub and thus publicly available. To accomplish this step,
    you should return to your web browser and go into the settings for your GitHub repository. Now, scroll down until
    you see the part of the screen with the label ``GitHub Pages'' and make sure that you are serving content from the
    master branch of your repository. You should now see a green checkmark and a label that gives the web site that is
    currently hosting your page. Can you find it? If not, then please ask for help from the instructor. Now,
    click on this link and then view the public version of your site!

  \item If you still have the ``{\tt bundle exec jekyll server}'' command running, you can shut it down by typing ``{\tt
    CTRL-c}'' in your terminal window. Now, start Atom and load the file called {\tt \_config.yml}. In this file, you
    should change the fields for the URL, title, and description to correspond to values that are most appropriate for
    your web site. In the section for the links in the navigation bar, you should (at least for now) delete all of them
    except for the one to ``{\tt aboutme}''. Finally, go to the bottom of this file and provide information for all of
    your social media accounts and your contact information. Please be sure that if you have, for instance, a Twitter
    account that you set the flag for this to {\tt true} in the configuration file.

  \item Now, please locate the ``About Me'' Markdown file that you completed in a previous assignment. Using Atom, copy
    the contents of this file to the clipboard. Next, locate the file called ``{\tt aboutme.md}'', delete the content in
    this file --- while taking care not to delete the header information at the top of the file --- and paste in the
    material from your previous assignment. To complete this step, you should also update the header so that it contains
    appropriate descriptive details for this file. Now, go back to your terminal window and run ``{\tt bundle exec
    jekyll server}'' and reload the locally-served content in your web browser.

  \item If you can see changes to the content, then you will have confirmation that your edits are working correctly.
    Before making this writing publicly available, you should carefully check it to ensure that it is compelling and
    error-free. After fixing any mistakes that you noticed, please use the ``{\tt git commit}'' and ``{\tt git push}''
    commands to ensure that your new content is also on the version of your site hosted by GitHub. Can you correctly
    view your new site?

  \item Finally, if you look in the GitHub repository for your new web site, you will notice that it still has the {\tt
    README.md} file that came with the original version of the ``Beautiful Jekyll'' template. You should delete the
    majority of the content in this file and add in some of your own original content. While it is appropriate for you
    to acknowledge that you created your web site with the help of this template, you should customize the {\tt
    README.md} with the steps that someone would need to take to download and fully use your new web site. You can use
    the content in this assignment sheet as a source for the content that you write in this file.

  \item After you have a working version of your web site, please paste the reference to the GitHub repository and to
    the web site into the {\tt \#web} channel of our Slack team. Then, please review the web site of at least one other
    member of the course. Once you find a way to improve your colleague's writing, please raise an issue in the issue
    tracker of their GitHub repository.

\end{enumerate}

\vspace*{-.175in}
\section*{Summary of the Required Deliverables}

This practical assignment invites you to complete the following tasks:

\vspace*{-.1in}
\begin{enumerate}
  \setlength{\itemsep}{0in}

  \item A cloned version of the ``Beautiful Jekyll'' template that is customized for your web site.
  \item A new ``About Me'' page that contains the revised version of your previous writing assignment.
  \item A customized version of the {\tt README.md} file that is in your web site's repository.
  \item An issue that you raised in the GitHub issue tracker for your colleague's web site.

\end{enumerate}
\vspace*{-.1in}

In adherence to the Honor Code, you should complete this practical assignment on an individual basis. While you may have
high-level conversations with others, any deliverables that are nearly identical to the work of others will be taken as
evidence of violating Allegheny College's \mbox{Honor Code}.

\end{document}
