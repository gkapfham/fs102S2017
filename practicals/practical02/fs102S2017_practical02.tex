% CS 111 style
% Typical usage (all UPPERCASE items are optional):
%       \input 111pre
%       \begin{document}
%       \MYTITLE{Title of document, e.g., Lab 1\\Due ...}
%       \MYHEADERS{short title}{other running head, e.g., due date}
%       \PURPOSE{Description of purpose}
%       \SUMMARY{Very short overview of assignment}
%       \DETAILS{Detailed description}
%         \SUBHEAD{if needed} ...
%         \SUBHEAD{if needed} ...
%          ...
%       \HANDIN{What to hand in and how}
%       \begin{checklist}
%       \item ...
%       \end{checklist}
% There is no need to include a "\documentstyle."
% However, there should be an "\end{document}."
%
%===========================================================
\documentclass[11pt,twoside,titlepage]{article}
%%NEED TO ADD epsf!!
\usepackage{threeparttop}
\usepackage{graphicx}
\usepackage{latexsym}
\usepackage{color}
\usepackage{listings}
\usepackage{fancyvrb}
%\usepackage{pgf,pgfarrows,pgfnodes,pgfautomata,pgfheaps,pgfshade}
\usepackage{tikz}
\usepackage[normalem]{ulem}
\tikzset{
    %Define standard arrow tip
%    >=stealth',
    %Define style for boxes
    oval/.style={
           rectangle,
           rounded corners,
           draw=black, very thick,
           text width=6.5em,
           minimum height=2em,
           text centered},
    % Define arrow style
    arr/.style={
           ->,
           thick,
           shorten <=2pt,
           shorten >=2pt,}
}
\usepackage[noend]{algorithmic}
\usepackage[noend]{algorithm}
\newcommand{\bfor}{{\bf for\ }}
\newcommand{\bthen}{{\bf then\ }}
\newcommand{\bwhile}{{\bf while\ }}
\newcommand{\btrue}{{\bf true\ }}
\newcommand{\bfalse}{{\bf false\ }}
\newcommand{\bto}{{\bf to\ }}
\newcommand{\bdo}{{\bf do\ }}
\newcommand{\bif}{{\bf if\ }}
\newcommand{\belse}{{\bf else\ }}
\newcommand{\band}{{\bf and\ }}
\newcommand{\breturn}{{\bf return\ }}
\newcommand{\mod}{{\rm mod}}
\renewcommand{\algorithmiccomment}[1]{$\rhd$ #1}
\newenvironment{checklist}{\par\noindent\hspace{-.25in}{\bf Checklist:}\renewcommand{\labelitemi}{$\Box$}%
\begin{itemize}}{\end{itemize}}
\pagestyle{threepartheadings}
\usepackage{url}
\usepackage{wrapfig}
% \usepackage{hyperref}
\usepackage[hidelinks]{hyperref}
%=========================
% One-inch margins everywhere
%=========================
\setlength{\topmargin}{0in}
\setlength{\textheight}{8.5in}
\setlength{\oddsidemargin}{0in}
\setlength{\evensidemargin}{0in}
\setlength{\textwidth}{6.5in}
%===============================
%===============================
% Macro for document title:
%===============================
\newcommand{\MYTITLE}[1]%
   {\begin{center}
     \begin{center}
     \bf
     FS 102 \\Software Everywhere\\
     Spring 2017\\
     \medskip
     \end{center}
     \bf
     #1
     \end{center}
}
%================================
% Macro for headings:
%================================
\newcommand{\MYHEADERS}[2]%
   {\lhead{#1}
    \rhead{#2}
    \immediate\write16{}
    \immediate\write16{DATE OF HANDOUT?}
    \read16 to \dateofhandout
    \lfoot{\sc Handed out on \dateofhandout}
    \immediate\write16{}
    \immediate\write16{HANDOUT NUMBER?}
    \read16 to\handoutnum
    \rfoot{Handout \handoutnum}
   }

%================================
% Macro for bold italic:
%================================
\newcommand{\bit}[1]{{\textit{\textbf{#1}}}}

%=========================
% Non-zero paragraph skips.
%=========================
\setlength{\parskip}{1ex}

%=========================
% Create various environments:
%=========================
\newcommand{\PURPOSE}{\par\noindent\hspace{-.25in}{\bf Purpose:\ }}
\newcommand{\SUMMARY}{\par\noindent\hspace{-.25in}{\bf Summary:\ }}
\newcommand{\DETAILS}{\par\noindent\hspace{-.25in}{\bf Details:\ }}
\newcommand{\HANDIN}{\par\noindent\hspace{-.25in}{\bf Hand in:\ }}
\newcommand{\SUBHEAD}[1]{\bigskip\par\noindent\hspace{-.1in}{\sc #1}\\}
%\newenvironment{CHECKLIST}{\begin{itemize}}{\end{itemize}}


\usepackage[compact]{titlesec}

\begin{document}
\MYTITLE{Practical 2\\Assigned: Wednesday, February 1, 2017\\Due: Wednesday, February 8, 2017 at the start of class\\``Checkmark'' grade}

\vspace*{-.2in}
\section*{Introduction}

Writers and presenters on the cutting-edge of technology often use a version control system to manage most of the
artifacts produced during the phases of drafting and delivering an article or a talk. In this course, we will always use
a GitHub repository to host the web site that will feature our writing and presentations. In this practical assignment,
you will learn how to create a preliminary version of your mobile-ready web site and take the first step towards adding
some of your existing content to that site. After finishing this assignment you should be able to view a local version
of your web site running on your development computer and a publicly available site that is hosted by GitHub.

\vspace*{-.1in}
\begin{itemize}
  \setlength{\itemsep}{-.01in}

\item {\bf If possible, use the laboratory computers.} If it is absolutely necessary for you to work on a different
  machine, be sure to regularly transfer your programs to the Alden machines and check their correctness. Please
  remember that, as stated in the syllabus, students should try to complete assignments using the specialized
  workstations in the laboratory. If you cannot use a laboratory computer, then please carefully explain the setup of
  your laptop to a teaching assistant or the course instructor when you are asking questions.

\item {\bf Follow each step carefully.} Slowly read each sentence in every assignment sheet, making sure that you
  precisely follow each instruction. Take notes about each step that you attempt, recording your questions and ideas
  and the challenges that you faced. If you are stuck, then please tell a teaching assistant or instructor what step
  you recently completed.

\item {\bf Regularly ask and answer questions.} Please log into Slack at the start of a class or practical session and
  then join the appropriate channel. If you have a question about one of the steps in an assignment, then you can post
  it to the designated channel. Or, you can ask a student sitting next to you or talk with a teaching assistant or the
  course instructor.

\item {\bf Store your files in Git}. Starting with this laboratory assignment, you will be responsible for storing all
  of your files in a Git repository. Please verify that you have saved your source code in your Git repository by
  typing ``{\tt git status}'' and ensuring everything is up to date.

\item {\bf Keep all of your files!} Don't delete your programs, output files, and reports after you hand them in---you
  will need them again later when you study for the quizzes and examinations and work on the other laboratory,
  practical, and final project assignments.

\item {\bf Back up your files regularly}. Use a flash drive, Google Drive, or your favorite backup method to keep a
  copy of your files in reserve. In the event of a system failure, you are responsible for ensuring that you have
  access to a recent backup copy of all your files.

% \item {\bf Review the Honor Code policy on the syllabus.} Remember that you may discuss programs with others, but
%   copying programs is a violation of the College's Honor Code.

\end{itemize}

\section*{Configuring Git and GitHub}

During this practical assignment and subsequent assignments, we will securely communicate with the GitHub.com servers
that will host all of our projects. In this practical assignment, we will perform all of the steps to configure the
accounts on the departmental servers and the GitHub service. Throughout this assignment, you should refer to the
following web site for additional information: \url{https://guides.github.com/activities/hello-world/}. As you will be
required to use Git in the remaining writing, speaking, and practical assignments and during the class sessions, please
be sure to keep a record of all of the steps that you complete and the challenges that you face. You are also
responsible for communicating with other students to ensure that everyone is able to successfully complete each of the
steps outlined in this assignment.

\begin{enumerate}

  \item If you do not already have a GitHub account, then please go to the GitHub web site and create one---make
    sure that you use your {\tt allegheny.edu} email address so that you can join the GitHub Educational Community as
    this step becomes necessary. Also, please make sure that you add a description of yourself and an appropriate
    professional photograph to your GitHub profile. For examples of what a professional GitHub profile might look like,
    please consider studying {\tt https://github.com/una} and {\tt https://github.com/gkapfham}.

  \item If you have never done so before, you must use the {\tt ssh-keygen} program to create secure-shell keys that you
    can use to support your communication with the GitHub servers. But, to start, this task requires you to type
    commands in a program that is known as a terminal. To run it, on the left side of your screen, click on the icon
    that contains the ``{\tt >}'' symbol. Alternatively, you can type the ``Super'' key, start typing the word
    ``terminal'', and then select that program. Another way to open a terminal involves typing the key combination {\tt
    <Ctrl>-<Alt>-t}.

  \item If you have not done so already, you will now need to run the {\tt ssh-keygen} command in your terminal window.
    Follow the prompts to create your keys and save them in the default directory (press ``Enter'' after you are
    prompted: ``{\tt Enter file in which to save the key ...  :}'', then press ``Enter'' twice if you do not wish to
    create a passphrase at this time or type your selected passphrase if you do). What files does {\tt ssh-keygen}
    produce? Where does this program store these files by default? Do you have questions about this step?

  \item Once you have created your ssh keys, you should raise your hand to invite either a teaching assistant or the
    course instructor to help you with the next steps. First, you must log into GitHub and look in the right corner for
    an account avatar with a down arrow. Click on this link and then select the ``Settings'' option. Now, scroll down
    until you found the ``SSH and GPG keys'' label on the left, click create a new ``SSH key'', and then upload your ssh
    key to GitHub. You can copy your to SSH key to the clipboard by going to the terminal and typing ``{\tt cat
    \textasciitilde{}/.ssh/id\_rsa.pub}'' command and then highlighting this output. When you are completing this step
    in your terminal window, please make sure that you only highlight the letters and numbers in your key---if you
    highlight any extra symbols or spaces then this step may not work correctly. Then, paste this into the text field in
    your web browser.

  \item Again, when you are completing these steps, please make sure that you take careful notes about the inputs,
    outputs, and behavior of each command. If there is something that you do not understand, then please ask the course
    instructor or the teaching assistant about it.

  \item Since this is your first practical assignment and you are still learning how to use the appropriate software,
    don't become frustrated if you make a mistake. Instead, use your mistakes as an opportunity for learning both about
    the necessary technology and the background and expertise of the other students in the class, the teaching
    assistants, and the course instructor. Remember, you can use Slack to talk with the instructor by using ``{\tt
    @gkapfham}'' in a channel.

\end{enumerate}

\vspace*{-.15in}
\section*{Critiquing the Writing on a Web Site}

In this next phase of this assignment, you should first go to the web site of the course instructor, which is available
at {\tt http://www.cs.allegheny.edu/sites/gkapfham/}. Next, you should browse this site so that you can locate the
GitHub repository that the instructor uses to store all of the source code for the web site. Now, you should take time
to use your web browser to look through the different directories of source code. What is source code? Can you find a
source code file and explain to another person in the class how it is connected to a page on the instructor's web site?

Using both a mobile device and either a desktop or laptop computer, you should read many of the pages on the course
instructor's web site. As you are doing this, you try to answer the questions from the ``Introduction'' slides in module
one on the course web site. Then, start to read this web site and identify mistakes in the writing, find ways to improve
its color, layout, and use of fonts, or assess whether the site is fully mobile-ready. Next, you should identify a
mistake or an enhancement that you would like to see the course instructor implement in a future version of the site.
Once you know what your ``issue'' is for the instructor's web site, go to its GitHub repository and click the ``Issues''
tab on the left side of the page. Using precise, error-free, and compelling language, use the GitHub issue tracker to
fully describe your issue concerning this site.

Once you have finished writing your issue, you should carefully review it using the ``Preview'' tab of the text area.
After you are confident that the writing is precise and error-free, please click the green ``Submit new issue'' button.
Next, you should read the issues that have been raised by the other students in the class and comment on at least one of
them. Do you agree with the points that are raised by your colleague? If you do, then add your persuasive argument as to
why this issue should be resolved in the next version of the site. If you disagree with a point made by another
issue-raiser, then persuasively argue why a different approach should be adopted instead.

\vspace*{-.05in}
\section*{Summary of the Required Deliverables}

This practical assignment invites you to complete the following tasks:

\vspace*{-.1in}
\begin{enumerate}
  \setlength{\itemsep}{0in}

  \item Create a professional GitHub profile that will host all of your writing and presentations.
  \item Upload your ssh key to the GitHub system, in support of completing the later assignments.
  \item Review the source code, design, layout, and writing on the course instructor's web site.
  \item Raise at least one publicly-visible issue on the GitHub repository for the instructor's web site.

\end{enumerate}
\vspace*{-.1in}

In adherence to the Honor Code, students should complete this assignment on an individual basis. While it is appropriate
for students in this class to have high-level conversations about the assignment, it is necessary to distinguish
carefully between the student who discusses the principles underlying a problem with others and the student who produces
assignments that are identical to, or merely variations on, someone else's work. Any deliverables that are nearly
identical to the work of others will be taken as evidence of violating Allegheny College's \mbox{Honor Code}.


\end{document}
