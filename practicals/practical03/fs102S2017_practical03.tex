% CS 111 style
% Typical usage (all UPPERCASE items are optional):
%       \input 111pre
%       \begin{document}
%       \MYTITLE{Title of document, e.g., Lab 1\\Due ...}
%       \MYHEADERS{short title}{other running head, e.g., due date}
%       \PURPOSE{Description of purpose}
%       \SUMMARY{Very short overview of assignment}
%       \DETAILS{Detailed description}
%         \SUBHEAD{if needed} ...
%         \SUBHEAD{if needed} ...
%          ...
%       \HANDIN{What to hand in and how}
%       \begin{checklist}
%       \item ...
%       \end{checklist}
% There is no need to include a "\documentstyle."
% However, there should be an "\end{document}."
%
%===========================================================
\documentclass[11pt,twoside,titlepage]{article}
%%NEED TO ADD epsf!!
\usepackage{threeparttop}
\usepackage{graphicx}
\usepackage{latexsym}
\usepackage{color}
\usepackage{listings}
\usepackage{fancyvrb}
%\usepackage{pgf,pgfarrows,pgfnodes,pgfautomata,pgfheaps,pgfshade}
\usepackage{tikz}
\usepackage[normalem]{ulem}
\tikzset{
    %Define standard arrow tip
%    >=stealth',
    %Define style for boxes
    oval/.style={
           rectangle,
           rounded corners,
           draw=black, very thick,
           text width=6.5em,
           minimum height=2em,
           text centered},
    % Define arrow style
    arr/.style={
           ->,
           thick,
           shorten <=2pt,
           shorten >=2pt,}
}
\usepackage[noend]{algorithmic}
\usepackage[noend]{algorithm}
\newcommand{\bfor}{{\bf for\ }}
\newcommand{\bthen}{{\bf then\ }}
\newcommand{\bwhile}{{\bf while\ }}
\newcommand{\btrue}{{\bf true\ }}
\newcommand{\bfalse}{{\bf false\ }}
\newcommand{\bto}{{\bf to\ }}
\newcommand{\bdo}{{\bf do\ }}
\newcommand{\bif}{{\bf if\ }}
\newcommand{\belse}{{\bf else\ }}
\newcommand{\band}{{\bf and\ }}
\newcommand{\breturn}{{\bf return\ }}
\newcommand{\mod}{{\rm mod}}
\renewcommand{\algorithmiccomment}[1]{$\rhd$ #1}
\newenvironment{checklist}{\par\noindent\hspace{-.25in}{\bf Checklist:}\renewcommand{\labelitemi}{$\Box$}%
\begin{itemize}}{\end{itemize}}
\pagestyle{threepartheadings}
\usepackage{url}
\usepackage{wrapfig}
% \usepackage{hyperref}
\usepackage[hidelinks]{hyperref}
%=========================
% One-inch margins everywhere
%=========================
\setlength{\topmargin}{0in}
\setlength{\textheight}{8.5in}
\setlength{\oddsidemargin}{0in}
\setlength{\evensidemargin}{0in}
\setlength{\textwidth}{6.5in}
%===============================
%===============================
% Macro for document title:
%===============================
\newcommand{\MYTITLE}[1]%
   {\begin{center}
     \begin{center}
     \bf
     FS 102 \\Software Everywhere\\
     Spring 2017\\
     \medskip
     \end{center}
     \bf
     #1
     \end{center}
}
%================================
% Macro for headings:
%================================
\newcommand{\MYHEADERS}[2]%
   {\lhead{#1}
    \rhead{#2}
    \immediate\write16{}
    \immediate\write16{DATE OF HANDOUT?}
    \read16 to \dateofhandout
    \lfoot{\sc Handed out on \dateofhandout}
    \immediate\write16{}
    \immediate\write16{HANDOUT NUMBER?}
    \read16 to\handoutnum
    \rfoot{Handout \handoutnum}
   }

%================================
% Macro for bold italic:
%================================
\newcommand{\bit}[1]{{\textit{\textbf{#1}}}}

%=========================
% Non-zero paragraph skips.
%=========================
\setlength{\parskip}{1ex}

%=========================
% Create various environments:
%=========================
\newcommand{\PURPOSE}{\par\noindent\hspace{-.25in}{\bf Purpose:\ }}
\newcommand{\SUMMARY}{\par\noindent\hspace{-.25in}{\bf Summary:\ }}
\newcommand{\DETAILS}{\par\noindent\hspace{-.25in}{\bf Details:\ }}
\newcommand{\HANDIN}{\par\noindent\hspace{-.25in}{\bf Hand in:\ }}
\newcommand{\SUBHEAD}[1]{\bigskip\par\noindent\hspace{-.1in}{\sc #1}\\}
%\newenvironment{CHECKLIST}{\begin{itemize}}{\end{itemize}}


\usepackage[compact]{titlesec}

\begin{document}
\MYTITLE{Practical 3\\Assigned: Wednesday, February 22, 2017\\Due: Wednesday, March 1, 2017 at the start of class\\``Checkmark'' grade}

\vspace*{-.2in}
\section*{Introduction}

Writers and presenters on the cutting-edge of technology often use a version control system to manage most of the
artifacts produced during the phases of drafting and delivering an article or a talk. In this practical assignment, you
will learn how to create a mobile-ready version of the slides for a presentation. The source code for your presentation
will be hosted in a GitHub repository and displayed by the RawGit content delivery network (CDN). After finishing this
assignment you should be able to view both a local version of your slides running on your development computer and a
publicly available version of the slides that are available in both GitHub and RawGit. As you are completing this
practical assignment, please make sure that you consider the following admonitions about using GitHub and RawGit to
complete a presentation assignment.

\vspace*{-.1in}
\begin{itemize}
  \setlength{\itemsep}{-.01in}

\item {\bf If possible, use the laboratory computers.} If it is absolutely necessary for you to work on a different
  machine, be sure to regularly transfer your presentation to the Alden machines and check their correctness. Please
  remember that, as stated in the syllabus, students should try to complete assignments using the specialized
  workstations in the laboratory. If you cannot use a laboratory computer, then, when you are asking questions, please
  carefully explain the setup of your laptop to a teaching assistant or to the course instructor.

\item {\bf Follow each step carefully.} Slowly read each sentence in every assignment sheet, making sure that you
  precisely follow each instruction. Take notes about each step that you attempt, recording your questions and ideas
  and the challenges that you faced. If you are stuck, then please tell a teaching assistant or instructor what step
  you recently completed.

\item {\bf Regularly ask and answer questions.} Please log into Slack at the start of a class or practical session and
  then join the appropriate channel. If you have a question about one of the steps in an assignment, then you can post
  it to the designated channel. Or, you can ask a student sitting next to you or talk with a teaching assistant or the
  course instructor.

\item {\bf Store your files in Git}. Starting with this laboratory assignment, you will be responsible for storing all
  of your files in a Git repository. Please verify that you have saved your source code in your Git repository by
  typing ``{\tt git status}'' and ensuring everything is up to date.

\item {\bf Keep all of your files!} Don't delete your programs, output files, and reports after you hand them in---you
  will need them again later when you study for the quizzes and examinations and work on the other laboratory,
  practical, and final project assignments.

\item {\bf Back up your files regularly}. Use a flash drive, Google Drive, or your favorite backup method to keep a
  copy of your files in reserve. In the event of a system failure, you are responsible for ensuring that you have
  access to a recent backup copy of all your files.

\end{itemize}

\section*{Configuring Git and GitHub}

As you complete the next part of this practical assignment please make sure that you follow each step carefully. If you
make a mistake in one of these steps it may require you to start over and follow all of the steps again. If you are not
sure how to do one of the requested actions, then please ask the course instructor or a teaching assistant. To start
this phase of the assignment, please visit and read the following web sites that contain the source code for two
presentations that were recently given by the course instructor:
\url{https://github.com/gkapfham/seke2015-panel-presentation} and
\url{https://github.com/gkapfham/townhall2016-presentation}. You should also study these web sites for the presentation
programming framework that we will adopt in this class: \url{http://lab.hakim.se/reveal-js/} and
\url{https://github.com/hakimel/reveal.js/}. Finally, you should learn about how the RawGit CDN works by visiting
\url{http://rawgit.com/}.

\begin{enumerate}

% Please see the course instructor or a teaching assistant if you do not know your GitHub user name (which should be, if
% possible, the same as your {\tt allegheny.edu} email user name) or you are not sure how to rename the repository.

  \item Your first task is to pick one of the presentation Git repositories created by the course instructor and fork
    it, using the knowledge and skills that you developed in the previous practical and laboratory assignments. Once you
    have forked your chosen repository repository, you need to go into its settings and rename it to ``{\tt
    fs102Spring2017-presentation1-<your GitHub user name>}''. The key task in this step is to create a presentation in a
    GitHub repository.

  \item Using SSH --- and not HTTPS! --- you should now clone the repository to your laboratory computer. You will do
    this by clicking the green ``Clone'' button, copying the address to the clipboard, and then typing ``{\tt git
    clone}'' and pasting the address in your terminal window. At this point, you should see the download of the source
    code for your new web site to your workstation. Ask the instructor for help if you think that this did not work
    correctly.

  \item Now, use the Atom text editor to explore the source code of your chosen presentation. For instance, if you
    picked the ``{\tt seke2015-panel-presentation}'' then you should look at the file called ``{\tt
    seke2015\_panel.html}''. Make sure that you understand all of the key features of this template (e.g., title slide,
    the use of color, and a background image). Next, you should delete all but one example of each type of slide and
    then customize each slide exemplar with content from a recent writing assignment. Of course, make sure that you have
    a title slide!

  \item Finally, update the {\tt README.md} file so that it introduces your presentation and then explains how to
    install and view it. You should test to make sure that your colleague can click on the title of your presentation
    and easily view your presentation in a web browser. Ultimately, you should have a simple presentation demonstrating
    that that you know how to program slides.

\end{enumerate}

\vspace*{-.175in}
\section*{Summary of the Required Deliverables}

This practical assignment invites you to complete the following tasks:

\vspace*{-.1in}
\begin{enumerate}
  \setlength{\itemsep}{0in}

  \item A cloned version of a presentation programmed by the course instructor, for use as a template.
  \item A customized version of the presentation that connects to any of the prior writing assignments.
  \item A customized version of the {\tt README.md} file that is in your web site's repository.

\end{enumerate}
\vspace*{-.1in}

In adherence to the Honor Code, you should complete this practical assignment on an individual basis. While you may have
high-level conversations with others, any deliverables that are nearly identical to the work of others will be taken as
evidence of violating Allegheny College's \mbox{Honor Code}.

\end{document}
