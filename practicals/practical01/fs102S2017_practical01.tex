% CS 111 style
% Typical usage (all UPPERCASE items are optional):
%       \input 111pre
%       \begin{document}
%       \MYTITLE{Title of document, e.g., Lab 1\\Due ...}
%       \MYHEADERS{short title}{other running head, e.g., due date}
%       \PURPOSE{Description of purpose}
%       \SUMMARY{Very short overview of assignment}
%       \DETAILS{Detailed description}
%         \SUBHEAD{if needed} ...
%         \SUBHEAD{if needed} ...
%          ...
%       \HANDIN{What to hand in and how}
%       \begin{checklist}
%       \item ...
%       \end{checklist}
% There is no need to include a "\documentstyle."
% However, there should be an "\end{document}."
%
%===========================================================
\documentclass[11pt,twoside,titlepage]{article}
%%NEED TO ADD epsf!!
\usepackage{threeparttop}
\usepackage{graphicx}
\usepackage{latexsym}
\usepackage{color}
\usepackage{listings}
\usepackage{fancyvrb}
%\usepackage{pgf,pgfarrows,pgfnodes,pgfautomata,pgfheaps,pgfshade}
\usepackage{tikz}
\usepackage[normalem]{ulem}
\tikzset{
    %Define standard arrow tip
%    >=stealth',
    %Define style for boxes
    oval/.style={
           rectangle,
           rounded corners,
           draw=black, very thick,
           text width=6.5em,
           minimum height=2em,
           text centered},
    % Define arrow style
    arr/.style={
           ->,
           thick,
           shorten <=2pt,
           shorten >=2pt,}
}
\usepackage[noend]{algorithmic}
\usepackage[noend]{algorithm}
\newcommand{\bfor}{{\bf for\ }}
\newcommand{\bthen}{{\bf then\ }}
\newcommand{\bwhile}{{\bf while\ }}
\newcommand{\btrue}{{\bf true\ }}
\newcommand{\bfalse}{{\bf false\ }}
\newcommand{\bto}{{\bf to\ }}
\newcommand{\bdo}{{\bf do\ }}
\newcommand{\bif}{{\bf if\ }}
\newcommand{\belse}{{\bf else\ }}
\newcommand{\band}{{\bf and\ }}
\newcommand{\breturn}{{\bf return\ }}
\newcommand{\mod}{{\rm mod}}
\renewcommand{\algorithmiccomment}[1]{$\rhd$ #1}
\newenvironment{checklist}{\par\noindent\hspace{-.25in}{\bf Checklist:}\renewcommand{\labelitemi}{$\Box$}%
\begin{itemize}}{\end{itemize}}
\pagestyle{threepartheadings}
\usepackage{url}
\usepackage{wrapfig}
% \usepackage{hyperref}
\usepackage[hidelinks]{hyperref}
%=========================
% One-inch margins everywhere
%=========================
\setlength{\topmargin}{0in}
\setlength{\textheight}{8.5in}
\setlength{\oddsidemargin}{0in}
\setlength{\evensidemargin}{0in}
\setlength{\textwidth}{6.5in}
%===============================
%===============================
% Macro for document title:
%===============================
\newcommand{\MYTITLE}[1]%
   {\begin{center}
     \begin{center}
     \bf
     FS 102 \\Software Everywhere\\
     Spring 2017\\
     \medskip
     \end{center}
     \bf
     #1
     \end{center}
}
%================================
% Macro for headings:
%================================
\newcommand{\MYHEADERS}[2]%
   {\lhead{#1}
    \rhead{#2}
    \immediate\write16{}
    \immediate\write16{DATE OF HANDOUT?}
    \read16 to \dateofhandout
    \lfoot{\sc Handed out on \dateofhandout}
    \immediate\write16{}
    \immediate\write16{HANDOUT NUMBER?}
    \read16 to\handoutnum
    \rfoot{Handout \handoutnum}
   }

%================================
% Macro for bold italic:
%================================
\newcommand{\bit}[1]{{\textit{\textbf{#1}}}}

%=========================
% Non-zero paragraph skips.
%=========================
\setlength{\parskip}{1ex}

%=========================
% Create various environments:
%=========================
\newcommand{\PURPOSE}{\par\noindent\hspace{-.25in}{\bf Purpose:\ }}
\newcommand{\SUMMARY}{\par\noindent\hspace{-.25in}{\bf Summary:\ }}
\newcommand{\DETAILS}{\par\noindent\hspace{-.25in}{\bf Details:\ }}
\newcommand{\HANDIN}{\par\noindent\hspace{-.25in}{\bf Hand in:\ }}
\newcommand{\SUBHEAD}[1]{\bigskip\par\noindent\hspace{-.1in}{\sc #1}\\}
%\newenvironment{CHECKLIST}{\begin{itemize}}{\end{itemize}}


\usepackage[compact]{titlesec}

\begin{document}
\MYTITLE{Practical 1\\Assigned: Friday, January 20, 2017\\Due: Wednesday, January 25, 2017 at the start of class\\``Checkmark'' grade}

\vspace*{-.2in}
\section*{Introduction}

Writers and presenters on the cutting-edge of technology often use a version control system to manage most of the
artifacts produced during the phases of drafting and delivering an article or a talk. In this course, we will always use
the Git distributed version control system to manage the files associated with our laboratory and practical assignments.

In this practical assignment, you will learn how to use the GitHub service for managing Git repositories and the {\tt
git} command-line tool in the Ubuntu operating system. Next, you will learn how to browse and start to understand the
source code of a web site, identify mistakes or enhancements that a web site needs, and then to bring these issues to
the attention of the site's creator. Please carefully adhere to the following guidelines for success when you are
completing this practical assignment and all of the subsequent writing, speaking, and practical assignments.

\vspace*{-.1in}
\begin{itemize}
  \setlength{\itemsep}{-.01in}

\item {\bf If possible, use the laboratory computers.} If it is absolutely necessary for you to work on a different
  machine, be sure to regularly transfer your programs to the Alden machines and check their correctness. Please
  remember that, as stated in the syllabus, students should try to complete assignments using the specialized
  workstations in the laboratory. If you cannot use a laboratory computer, then please carefully explain the setup of
  your laptop to a teaching assistant or the course instructor when you are asking questions.

\item {\bf Follow each step carefully.} Slowly read each sentence in every assignment sheet, making sure that you
  precisely follow each instruction. Take notes about each step that you attempt, recording your questions and ideas
  and the challenges that you faced. If you are stuck, then please tell a teaching assistant or instructor what step
  you recently completed.

\item {\bf Regularly ask and answer questions.} Please log into Slack at the start of a laboratory or practical
  session and then join the appropriate channel. If you have a question about one of the steps in an assignment, then
  you can post it to the designated channel. Or, you can ask a student sitting next to you or talk with a teaching
  assistant or the course instructor.

\item {\bf Store your files in Git}. Starting with this laboratory assignment, you will be responsible for storing all
  of your files in a Git repository. Please verify that you have saved your source code in your Git repository by
  typing ``{\tt git status}'' and ensuring everything is up to date.

\item {\bf Keep all of your files!} Don't delete your programs, output files, and reports after you hand them in---you
  will need them again later when you study for the quizzes and examinations and work on the other laboratory,
  practical, and final project assignments.

\item {\bf Back up your files regularly}. Use a flash drive, Google Drive, or your favorite backup method to keep a
  copy of your files in reserve. In the event of a system failure, you are responsible for ensuring that you have
  access to a recent backup copy of all your files.

% \item {\bf Review the Honor Code policy on the syllabus.} Remember that you may discuss programs with others, but
%   copying programs is a violation of the College's Honor Code.

\end{itemize}

\section*{Configuring Git and GitHub}

During this practical assignment and subsequent assignments, we will securely communicate with the GitHub.org servers
that will host all of our projects.  In this practical assignment, we will perform all of the steps to configure the
accounts on the departmental servers and the GitHub service.  Throughout the assignment, you should refer to the
following web site for additional information: \url{https://guides.github.com/activities/hello-world/}. As you will be
required to use Git in the remaining writing, speaking, and practical assignments and during the class sessions, please
be sure to keep a record of all of the steps that you complete and the challenges that you face. You are also
responsible for communicating with a partner to ensure that each of you is able to successfully complete each of the
steps outlined in this assignment.

\begin{enumerate}

  \item If you do not already have a GitHub account, then please go to the GitHub web site and create one---make
    sure that you use your {\tt allegheny.edu} email address so that you can join the GitHub Educational Community as
    this step becomes necessary. Also, please make sure that you add a description of yourself and an appropriate
    professional photograph to your GitHub profile. For examples of what a professional GitHub profile might look like,
    please consider studying {\tt https://github.com/una} and {\tt https://github.com/gkapfham}.

  \item If you have never done so before, you must use the {\tt ssh-keygen} program to create secure-shell keys that you
    can use to support your communication with the GitHub servers. Follow the prompts to create your keys and save
    them in the default directory (press ``Enter'' after you are prompted: ``{\tt Enter file in which to save the key
    ...  :}'', then press ``Enter'' twice if you do not wish to create a passphrase at this time or type your selected
    passphrase if you do). Type {\tt man ssh-keygen} and talk with your partner to learn more about this program. What
    files does {\tt ssh-keygen} produce? Where does this program store these files by default?

  \item Once you have created your ssh keys, you should raise your hand to invite either a teaching assistant or the
    course instructor to help you with the next steps. First, you must log into GitHub and look in the right corner for
    an account avatar with a down arrow. Click on this blue link and then select the ``GitHub Settings'' option. Now,
    scroll down until you found the ``SSH keys'' option and upload your ssh key to GitHub. You can copy your SSH key by
    going to the terminal and typing ``{\tt cat \textasciitilde{}/.ssh/id\_rsa.pub}'' command.

\end{enumerate}

\section*{Critiquing the Writing on a Web Site}

Since this is your first practical assignment and you are still learning how to use the appropriate hardware and
software, don't become frustrated if you make a mistake. Instead, use your mistakes as an opportunity for learning both
about the necessary technology and the background and expertise of the other students in the class, the teaching
assistants, and the course instructor. Remember, you can use Slack to talk with the instructor by using ``{\tt
@gkapfham}'' in a channel.

\end{document}
