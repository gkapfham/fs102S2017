\input{102pre.tex}

\usepackage[compact]{titlesec}

\begin{document}

\MYTITLE{Assignment 6\\Assigned: Wednesday, April 5, 2017\\Due: Monday, April 10, 2017 \\ Randomly ordered
presentations will start on the assignment's due date}

\vspace*{-.2in}
\section*{Giving Your Second Presentation}

Now that you have begun to investigate the topic of digital piracy and to collect evidence-based ideas and opinions
about this topic, you are ready to present your own views on this issue. Along with having a captivating title and an
attention-getting device, your presentation should also contain illustrative data points, screenshots, and/or concrete
examples. In particular, at least one of your slides should contain a research-based finding that you derived from a
research article.

Moreover, your presentation should feature an opinion that you discovered during the interview(s) that you conducted for
a previous writing assignment. Finally, during your presentation you should leverage at least one of the ``templates''
provided in the {\em They Say/I Say} book. For instance, you might use a template such a ``At the same time that I
believe $\rule{1cm}{0.15mm}$, I also believe $\rule{1cm}{0.15mm}$.'' or ``The standard way of thinking about digital
piracy is that $\rule{1cm}{0.15mm}$''. Since your presentation involves you entering an ongoing debate about a
controversial topic, you may also consider using templates like ``In discussions of digital piracy, one controversial
issue has been $\rule{1cm}{0.15mm}$. On the one hand, $\rule{1cm}{0.15mm}$ argues $\rule{1cm}{0.15mm}$. On the other
hand, $\rule{1cm}{0.15mm}$ contends $\rule{1cm}{0.15mm}$. My own view on digital piracy is $\rule{1cm}{0.15mm}$.'' As
suggested in the {\em Demystifying Public Speaking} book, you should ensure that your presentation has ``takeaways''
that are easy for the audience to recognize and understand.

After picking a partner with whom you will have an ``editor-presenter'' relationship, you should create a new GitHub
repository for your presentation using a similar naming convention to the one introduced in the previous presentation
assignment. The source code for your presentation will be hosted in a new GitHub repository and displayed by the RawGit
content delivery network (CDN). After finishing this assignment you should be able to view both a local version of your
slides running on your development computer and a publicly visible version of the slides that are available in both
GitHub and RawGit. Finally, your slides should contain enough content to ensure that they support you giving a talk that
lasts for between three and five minutes.

Your focus should be on the preparation of simple slides that help you to make your main point. If your schedule
permits, you are welcome to enhance the slide template that you created in the previous practical assignment by, for
instance, changing the background and/or highlight colors and picking new fonts. Once you have finished preparing your
slides, you should ensure that your editor gives you feedback on the content of each slide and the flow between the
slides. After ensuring that you have resolved all of the concerns raised by your editor, you should also practice giving
the presentation to your editor and your colleagues not in this class. You should deliver your practice talks in our
classroom, further ensuring that you can quickly locate and display \mbox{your slides}.

Once you have finished your presentation slides, you should create a link to them on your web site. Along with giving
the title of your presentation, you should include the date on which your talk will be given and, optionally, an image
from or a screenshot of one of your slides.

\vspace*{-.1in}
\section*{Summary of the Required Deliverables}

This writing assignment invites you to complete the following deliverables. After completing the assignment, please make
sure that you submit a printed version of these items to the instructor.

\vspace*{-.1in}
\begin{enumerate}
  \setlength{\itemsep}{-.01in}

  \item Presentation slides, hosted in a separate GitHub repository, suitable for a five-minute talk.
  \item Evidence in the GitHub issue tracker for the frequent discussion and revision of your slides.
  \item Reviews, through multiple GitHub issues and a full-featured discussion, of your peers' slides.
  \item An entry in the listing of presentations on your web site for your new presentation.

\end{enumerate}
\vspace*{-.1in}

In adherence to the Honor Code, students should complete the main parts of this assignment on an individual basis. While
you can and should incorporate feedback from your editor, any deliverables that are nearly identical to the work of
others will be taken as evidence of violating Allegheny College's \mbox{Honor Code}. Please see the course instructor
with questions about this policy.

\end{document}
