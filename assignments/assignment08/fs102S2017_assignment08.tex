% CS 111 style
% Typical usage (all UPPERCASE items are optional):
%       \input 111pre
%       \begin{document}
%       \MYTITLE{Title of document, e.g., Lab 1\\Due ...}
%       \MYHEADERS{short title}{other running head, e.g., due date}
%       \PURPOSE{Description of purpose}
%       \SUMMARY{Very short overview of assignment}
%       \DETAILS{Detailed description}
%         \SUBHEAD{if needed} ...
%         \SUBHEAD{if needed} ...
%          ...
%       \HANDIN{What to hand in and how}
%       \begin{checklist}
%       \item ...
%       \end{checklist}
% There is no need to include a "\documentstyle."
% However, there should be an "\end{document}."
%
%===========================================================
\documentclass[11pt,twoside,titlepage]{article}
%%NEED TO ADD epsf!!
\usepackage{threeparttop}
\usepackage{graphicx}
\usepackage{latexsym}
\usepackage{color}
\usepackage{listings}
\usepackage{fancyvrb}
%\usepackage{pgf,pgfarrows,pgfnodes,pgfautomata,pgfheaps,pgfshade}
\usepackage{tikz}
\usepackage[normalem]{ulem}
\tikzset{
    %Define standard arrow tip
%    >=stealth',
    %Define style for boxes
    oval/.style={
           rectangle,
           rounded corners,
           draw=black, very thick,
           text width=6.5em,
           minimum height=2em,
           text centered},
    % Define arrow style
    arr/.style={
           ->,
           thick,
           shorten <=2pt,
           shorten >=2pt,}
}
\usepackage[noend]{algorithmic}
\usepackage[noend]{algorithm}
\newcommand{\bfor}{{\bf for\ }}
\newcommand{\bthen}{{\bf then\ }}
\newcommand{\bwhile}{{\bf while\ }}
\newcommand{\btrue}{{\bf true\ }}
\newcommand{\bfalse}{{\bf false\ }}
\newcommand{\bto}{{\bf to\ }}
\newcommand{\bdo}{{\bf do\ }}
\newcommand{\bif}{{\bf if\ }}
\newcommand{\belse}{{\bf else\ }}
\newcommand{\band}{{\bf and\ }}
\newcommand{\breturn}{{\bf return\ }}
\newcommand{\mod}{{\rm mod}}
\renewcommand{\algorithmiccomment}[1]{$\rhd$ #1}
\newenvironment{checklist}{\par\noindent\hspace{-.25in}{\bf Checklist:}\renewcommand{\labelitemi}{$\Box$}%
\begin{itemize}}{\end{itemize}}
\pagestyle{threepartheadings}
\usepackage{url}
\usepackage{wrapfig}
% \usepackage{hyperref}
\usepackage[hidelinks]{hyperref}
%=========================
% One-inch margins everywhere
%=========================
\setlength{\topmargin}{0in}
\setlength{\textheight}{8.5in}
\setlength{\oddsidemargin}{0in}
\setlength{\evensidemargin}{0in}
\setlength{\textwidth}{6.5in}
%===============================
%===============================
% Macro for document title:
%===============================
\newcommand{\MYTITLE}[1]%
   {\begin{center}
     \begin{center}
     \bf
     FS 102 \\Software Everywhere\\
     Spring 2017\\
     \medskip
     \end{center}
     \bf
     #1
     \end{center}
}
%================================
% Macro for headings:
%================================
\newcommand{\MYHEADERS}[2]%
   {\lhead{#1}
    \rhead{#2}
    \immediate\write16{}
    \immediate\write16{DATE OF HANDOUT?}
    \read16 to \dateofhandout
    \lfoot{\sc Handed out on \dateofhandout}
    \immediate\write16{}
    \immediate\write16{HANDOUT NUMBER?}
    \read16 to\handoutnum
    \rfoot{Handout \handoutnum}
   }

%================================
% Macro for bold italic:
%================================
\newcommand{\bit}[1]{{\textit{\textbf{#1}}}}

%=========================
% Non-zero paragraph skips.
%=========================
\setlength{\parskip}{1ex}

%=========================
% Create various environments:
%=========================
\newcommand{\PURPOSE}{\par\noindent\hspace{-.25in}{\bf Purpose:\ }}
\newcommand{\SUMMARY}{\par\noindent\hspace{-.25in}{\bf Summary:\ }}
\newcommand{\DETAILS}{\par\noindent\hspace{-.25in}{\bf Details:\ }}
\newcommand{\HANDIN}{\par\noindent\hspace{-.25in}{\bf Hand in:\ }}
\newcommand{\SUBHEAD}[1]{\bigskip\par\noindent\hspace{-.1in}{\sc #1}\\}
%\newenvironment{CHECKLIST}{\begin{itemize}}{\end{itemize}}


\usepackage[compact]{titlesec}

\begin{document}

\MYTITLE{Final Project Assignment\\Assigned: Monday, April 24, 2017\\Due: May 9, 2017 no later than 5:00 pm}

\vspace*{-.2in}
\section*{Introduction}

Leveraging the feedback that they received for each of their writing and presentation assignments, you will prepare a
final project portfolio, available as a mobile-ready web site hosted by GitHub, that showcases the revised version of
each writing and presentation assignment. In addition, you will prepare a final article, of at least 1000 words, that
draws on all of your past articles and presentations while still persuasively articulating your view of a course-related
issue. Moreover, it should identify some challenge concerning how people in society use modern technology and then
persuasively articulate a solution to this issue. Your article should leverage all of the writing techniques in the
following course textbooks to persuasively argue your point. In particular, your article should employ at least one of
the ``templates'' in the {\em They Say, I Say} book to persuasively explain why you have proposed an appropriate
solution to your chosen problem.

\noindent{\em A Writer's Reference}. Nancy Sommers. Seventh Edition, ISBN-10: 0312601433, ISBN-13: 978-0312601430,
245 pages, 2010. (From your First-Year/Sophomore 101 course).

\noindent{\em They Say, I Say: The Moves That Matter in Academic Writing}. Gerald Graff and Cathy Birkenstein. Second Edition,
ISBN-10: 039393361X, ISBN-13: 978-0393933611, 245 pages, 2010.

\noindent{\em BUGS in Writing: A Guide to Debugging Your Prose}. Lyn Dupr\'e. Second Edition, ISBN-10: 020137921X,
ISBN-13: 978-0201379211, 704 pages, 1998.

% \vspace*{-.15in}
\section*{Summary of the Required Deliverables}

This writing assignment invites you to complete the following deliverables. After completing the assignment, please make
sure that you submit a printed version of these items to the instructor.

\vspace*{-.1in}
\begin{enumerate}
  \setlength{\itemsep}{-.01in}

  \item An at least 1000 word article stating a challenge and solution in any area of modern technology.
  \item A header image from Flickr that is released under a Creative Commons licensing agreement.
  \item Evidence in the GitHub issue tracker for the frequent discussion and revision of your article.
  \item Documentation of a social media campaign that convinces your friends to read your article.
  \item Reviews, through multiple GitHub issues and a full-featured discussion, of your peers' writing.
  \item Accounting for all previous feedback, a fully revised version of your mobile-ready web site.

\end{enumerate}
\vspace*{-.1in}

% The Academic Honor Program that governs the entire academic program at Allegheny College is described in the Allegheny
% Course Catalogue. The Honor Program applies to all work that is submitted for academic credit or to meet non-credit
% requirements for graduation at Allegheny College. This includes all work assigned for this class (e.g., examinations,
% writing assignments, and presentations). All students who have enrolled in the College will work under the Honor
% Program. Each student who has matriculated at the College has acknowledged the following pledge:

% \vspace*{-.1in}
% \begin{quote}
%   I hereby recognize and pledge to fulfill my responsibilities, as defined in the Honor Code, and to maintain the
%   integrity of both myself and the College community as a whole.
% \end{quote}
% \vspace*{-.1in}

In adherence to the Honor Code, students should complete the main writing of this assignment on an individual basis.
While you can and should incorporate feedback from your editor, any deliverables that are nearly identical to the work
of others will be taken as evidence of violating Allegheny College's \mbox{Honor Code}. Please see the course instructor
with questions about this policy.

\end{document}
