% CS 111 style
% Typical usage (all UPPERCASE items are optional):
%       \input 111pre
%       \begin{document}
%       \MYTITLE{Title of document, e.g., Lab 1\\Due ...}
%       \MYHEADERS{short title}{other running head, e.g., due date}
%       \PURPOSE{Description of purpose}
%       \SUMMARY{Very short overview of assignment}
%       \DETAILS{Detailed description}
%         \SUBHEAD{if needed} ...
%         \SUBHEAD{if needed} ...
%          ...
%       \HANDIN{What to hand in and how}
%       \begin{checklist}
%       \item ...
%       \end{checklist}
% There is no need to include a "\documentstyle."
% However, there should be an "\end{document}."
%
%===========================================================
\documentclass[11pt,twoside,titlepage]{article}
%%NEED TO ADD epsf!!
\usepackage{threeparttop}
\usepackage{graphicx}
\usepackage{latexsym}
\usepackage{color}
\usepackage{listings}
\usepackage{fancyvrb}
%\usepackage{pgf,pgfarrows,pgfnodes,pgfautomata,pgfheaps,pgfshade}
\usepackage{tikz}
\usepackage[normalem]{ulem}
\tikzset{
    %Define standard arrow tip
%    >=stealth',
    %Define style for boxes
    oval/.style={
           rectangle,
           rounded corners,
           draw=black, very thick,
           text width=6.5em,
           minimum height=2em,
           text centered},
    % Define arrow style
    arr/.style={
           ->,
           thick,
           shorten <=2pt,
           shorten >=2pt,}
}
\usepackage[noend]{algorithmic}
\usepackage[noend]{algorithm}
\newcommand{\bfor}{{\bf for\ }}
\newcommand{\bthen}{{\bf then\ }}
\newcommand{\bwhile}{{\bf while\ }}
\newcommand{\btrue}{{\bf true\ }}
\newcommand{\bfalse}{{\bf false\ }}
\newcommand{\bto}{{\bf to\ }}
\newcommand{\bdo}{{\bf do\ }}
\newcommand{\bif}{{\bf if\ }}
\newcommand{\belse}{{\bf else\ }}
\newcommand{\band}{{\bf and\ }}
\newcommand{\breturn}{{\bf return\ }}
\newcommand{\mod}{{\rm mod}}
\renewcommand{\algorithmiccomment}[1]{$\rhd$ #1}
\newenvironment{checklist}{\par\noindent\hspace{-.25in}{\bf Checklist:}\renewcommand{\labelitemi}{$\Box$}%
\begin{itemize}}{\end{itemize}}
\pagestyle{threepartheadings}
\usepackage{url}
\usepackage{wrapfig}
% \usepackage{hyperref}
\usepackage[hidelinks]{hyperref}
%=========================
% One-inch margins everywhere
%=========================
\setlength{\topmargin}{0in}
\setlength{\textheight}{8.5in}
\setlength{\oddsidemargin}{0in}
\setlength{\evensidemargin}{0in}
\setlength{\textwidth}{6.5in}
%===============================
%===============================
% Macro for document title:
%===============================
\newcommand{\MYTITLE}[1]%
   {\begin{center}
     \begin{center}
     \bf
     FS 102 \\Software Everywhere\\
     Spring 2017\\
     \medskip
     \end{center}
     \bf
     #1
     \end{center}
}
%================================
% Macro for headings:
%================================
\newcommand{\MYHEADERS}[2]%
   {\lhead{#1}
    \rhead{#2}
    \immediate\write16{}
    \immediate\write16{DATE OF HANDOUT?}
    \read16 to \dateofhandout
    \lfoot{\sc Handed out on \dateofhandout}
    \immediate\write16{}
    \immediate\write16{HANDOUT NUMBER?}
    \read16 to\handoutnum
    \rfoot{Handout \handoutnum}
   }

%================================
% Macro for bold italic:
%================================
\newcommand{\bit}[1]{{\textit{\textbf{#1}}}}

%=========================
% Non-zero paragraph skips.
%=========================
\setlength{\parskip}{1ex}

%=========================
% Create various environments:
%=========================
\newcommand{\PURPOSE}{\par\noindent\hspace{-.25in}{\bf Purpose:\ }}
\newcommand{\SUMMARY}{\par\noindent\hspace{-.25in}{\bf Summary:\ }}
\newcommand{\DETAILS}{\par\noindent\hspace{-.25in}{\bf Details:\ }}
\newcommand{\HANDIN}{\par\noindent\hspace{-.25in}{\bf Hand in:\ }}
\newcommand{\SUBHEAD}[1]{\bigskip\par\noindent\hspace{-.1in}{\sc #1}\\}
%\newenvironment{CHECKLIST}{\begin{itemize}}{\end{itemize}}


\usepackage[compact]{titlesec}

\begin{document}

\MYTITLE{Assignment 5\\Assigned: Friday, April 21, 2017\\Due: May 1, 2017 at the start of class}

\vspace*{-.2in}
\section*{Introduction}

Now that you have had more practice in persuasive writing, you are ready to identify a problem in a field, propose a
solution to this problem, and then anticipate and respond to the objections to you solution. In this assignment, you
will identify a problem in the field of academic plagiarism and then articulate and defend a solution to the problem. In
particular, you writing should use ``They Say/I Say'' templates to entertain objections and/or name naysayers.
Additionally, your article will have both a header image that is released through Flickr under a Creative Commons
License and a license statement for the image. Along with having a captivating title, a TL;DR, and links to referenced
articles and web sites, your second feature article will be promoted through a social media campaign. As you complete
this assignment, please adhere to the following guidelines.

\vspace*{-.1in}
\begin{itemize}
  \setlength{\itemsep}{-.01in}

\item {\bf If possible, use the laboratory computers.} If it is absolutely necessary for you to work on a different
  machine, be sure to regularly transfer your articles to the Alden machines and check their correctness. Please
  remember that, as stated in the syllabus, students should try to complete assignments using the specialized
  workstations in the laboratory. If you cannot use a laboratory computer, then please carefully explain the setup of
  your laptop to a teaching assistant or the course instructor when you are asking questions.

\item {\bf Follow each step carefully.} Slowly read each sentence in every assignment sheet, making sure that you
  precisely follow each instruction. Take notes about each step that you attempt, recording your questions and ideas
  and the challenges that you faced. If you are stuck, then please tell a teaching assistant or instructor what step
  you recently completed.

\item {\bf Regularly ask and answer questions.} Please log into Slack at the start of a class or practical session and
  then join the appropriate channel. If you have a question about one of the steps in an assignment, then you can post
  it to the designated channel. Or, you can ask a student sitting next to you or talk with a teaching assistant or the
  course instructor.

\item {\bf Store your files in Git}. Starting with this laboratory assignment, you will be responsible for storing all
  of your files in a Git repository. Please verify that you have saved your source code in your Git repository by
  typing ``{\tt git status}'' and ensuring everything is up to date.

\item {\bf Keep all of your files!} Don't delete your programs, output files, and reports after you hand them in---you
  will need them again later when you study for the quizzes and examinations and work on the other laboratory,
  practical, and final project assignments.

\item {\bf Back up your files regularly}. Use a flash drive, Google Drive, or your favorite backup method to keep a
  copy of your files in reserve. In the event of a system failure, you are responsible for ensuring that you have
  access to a recent backup copy of all your files.

\end{itemize}

\section*{Publishing a ``Portrait of Piracy''}

This assignment invites you to complete a ``portrait of piracy''. You are responsible for interviewing individual(s) who
can share an interesting, informative, and balanced account about software, music, and/or movie piracy. For this
assignment, you may want to interview people who are strongly in support of or opposed to piracy. Students are
encouraged to pick interview candidates who may hold an opinion that is different from their own. Before conducting your
interview, please carefully think about the people who will answer your questions and then prepare a detailed listing of
the issues --- and at least ten questions --- that you want to discuss during your time together.

Once the interview(s) are finished and you have recorded all of the responses to your questions, you should prepare a
written report of the interview(s). Going beyond a simple restatement of the questions and the answers, this article
should include personal insights, additional facts and details, counterpoints to stated opinions, and other content that
would yield a nuanced portrait of modern digital piracy. In particular, you should ensure that at least one of the
questions you ask the interviewee is derived from a research-based finding from one of the assigned papers.

In summary, this assignment invites you to publish an article that exposes, using concrete and compelling examples
derived from the assigned articles and the responses of the interviewee, the current state of digital piracy. After
picking a partner with whom you will have an ``editor-writer'' relationship, please publish a high-quality and
error-free article by following these steps.

\vspace*{-.1in}

\begin{enumerate}

  \itemsep 0em

  \item As in the last writing assignment, you should again pick a partner. You will serve as the ``editor'' for your
    partner's writing and your partner will be the ``editor'' for your writing. To start, please make sure that you and
    your partner know each other's GitHub site and web site. An editor will provide feedback through the issue tracker
    by clicking the ``Commits'' tab in GitHub's main interface, finding the commit number that contains the issue that
    you have identified in the writing, copying this number to the clipboard, and then using it when you report the
    issue. A writer will then consider the feedback, respond to it, and, when appropriate, resolve the issue that the
    editor raised. While it is acceptable to read and mark printed versions of the article, all discussion should be
    publicly visible in the issue tracker.

  \item Using your terminal window, please go to the directory that contains the files for your web site. Now, find and
    change into the {\tt \_posts} directory where you will find the sample posts that come with the template that we are
    using for your web site. Using your previous file as an example, create a new file that you can use to store your
    writing for this assignment. That is, you should use the file browser (or, the ``{\tt mv}'' command if you have
    experience with it) to rename a copy of the past file to one that would be appropriate for your new article. Please
    consider naming your file according to the title that you recently developed.

  \item At this point, you should use the Atom text editor to change the contents of the file that you renamed in the
    previous step. The first thing that you should do is revise the ``header'' of this file that is at the very top of
    the editor's window. For instance, you should create a new title and subtitle for your article. Next, delete the
    rest of the content from the prior article.

  \item Your article must feature a header image that you download from Flickr. Please make sure that your image is
    licensed under a Creative Commons license and is suitable for use in an academic setting. To start your search for a
    header image, please go to the following web site: {\tt https://www.flickr.com/creativecommons/}. Once you have
    found the image that you want to include on your site, you can use the ``Flickr attribution helper'' that is
    available at {\tt https://cogdog.github.io/flickr-cc-helper/} to get the correct reference for your image. Next,
    download your image and save it in the {\tt img/} directory of your web site. Ultimately, the image should appear at
    the top of your article. You should also make sure that the full license statement and the credit for the image is
    at the bottom of the article. Please see the instructor or a teaching assistant if you have questions about the
    completion of this step.

  \item After you finish modifying the header of this file you should use the ``{\tt git add}'' and ``{\tt git commit}''
    and ``{\tt git push}'' commands to transfer your content to the GitHub servers. You should also use the ``{\tt
    bundle exec jekyll serve}'' command to preview a local version of your web page. Now, you are ready to start
    drafting a version of your article! As you are writing, please make sure that you write and revise in an incremental
    fashion, committing and pushing your sentences on a regular basis --- generally, small modifications are better than
    large ones because they make it easier for you and your editor to track and comment on your developing article. With
    that said, please recall that once you run the ``{\tt git push}'' command your content will be ``live'' on your web
    site. Students who want to explore a potentially better way to write that supports repeated pushing of content are
    encouraged to review the ``GitHub Flow'' handout and talk with the course instructor about using Git branches.

  \item Since this assignment invites you to ask your chosen interviewee about at least one research-based finding, you
    should make sure that you have read the assigned papers for this module. In particular, you should concentrate on
    reading and understanding ``Adolescent Self-Control and Music and Movie Piracy'' by Malin and Fowers and ``Is Music
    Downloading the New Prohibition? What Students Reveal through an Ethical Dilemma'' by Altschuller and Benbunan-Fich.
    You are also encourage to read and cite other research articles that are related to the topic of software, music,
    and movie piracy. Please see the course instructor during office hours if you have questions about any of the
    findings in these research articles.

  \item Instead of waiting to get feedback from your editor when you have finished the entire article, you are
    encouraged to write a draft of your content, publish it on your web site, and then request feedback from your
    editor. For instance, it might be a good idea to ask your editor to review the questions that you plan to ask the
    interviewee. You can also write about the research-based finding that you plan to articulate in your article. Once
    you have taken some of these steps, you can use the GitHub issue tracker to have a conversation with your editor
    about the best way to improve your article. You and your editor should use the issue tracker to have a thorough
    discussion about the strengths and weaknesses of your writing. Please note that your critique is not limited to the
    words in the article itself! You should also provide feedback about the structure of the article, the header image
    and TL;DR, the citation to the header image, and other aspect's of the article's content and layout.

  % \item As you are writing, you may wonder how many words are currently in your article. If your file is called ``{\tt
  %   article.md}'', then you can type ``{\tt wc -w article.md}'' in your terminal window to count the number of words in
  %   your article. While this count may be slightly inaccurate, the use of this ``word count'' program is a good way to
  %   estimate the length of your article.

  \item Even though we will work on this assignment during at least two class sessions, you should budget enough time to
    outline, draft, discuss, and frequently revise this writing assignment. Remember, your focus should be on publishing
    an article that is concise, accessible, and error-free. Also, the main text of this writing assignment should be
    accompanied by a ``TL;DR'' or a ``Too Long; Didn't Read'' that summarizes the key points about the writing. Please
    see the instructor if you confront difficulties when using GitHub or if you have additional questions.

  % \item Since this is the second assignment in which you will publish a complete article, don't become frustrated if you
  %   make a mistake. Instead, use your mistakes as an opportunity for learning both about the software technology that we
  %   use for writing and the background and expertise of the other students in the class, the teaching assistants, and
  %   the course instructor.

\end{enumerate}

\vspace*{-.15in}
\section*{Summary of the Required Deliverables}

This writing assignment invites you to complete the following deliverables. After completing the assignment, please make
sure that you submit a printed version of these items to the instructor.

\vspace*{-.1in}
\begin{enumerate}
  \setlength{\itemsep}{-.01in}

  \item A feature-length interview that furnishes an evidence-based portrait of modern digital piracy.
  \item A header image from Flickr that is released under a Creative Commons licensing agreement.
  \item Evidence in the GitHub issue tracker for the frequent discussion and revision of your article.
  \item Documentation of a social media campaign that convinces your friends to read your article.
  \item Reviews, through multiple GitHub issues and a full-featured discussion, of your peers' writing.

\end{enumerate}
\vspace*{-.1in}

% While it is appropriate for students in this class to have high-level conversations about the assignment, it is
% necessary to distinguish carefully between the student who discusses the principles underlying a problem with others and
% the student who produces assignments that are identical to, or merely variations on, someone else's work.

In adherence to the Honor Code, students should complete the main writing of this assignment on an individual basis.
While you can and should incorporate feedback from your editor, any deliverables that are nearly identical to the work
of others will be taken as evidence of violating Allegheny College's \mbox{Honor Code}. Please see the course instructor
with questions about this policy.

\end{document}
