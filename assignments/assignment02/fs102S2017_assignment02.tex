% CS 111 style
% Typical usage (all UPPERCASE items are optional):
%       \input 111pre
%       \begin{document}
%       \MYTITLE{Title of document, e.g., Lab 1\\Due ...}
%       \MYHEADERS{short title}{other running head, e.g., due date}
%       \PURPOSE{Description of purpose}
%       \SUMMARY{Very short overview of assignment}
%       \DETAILS{Detailed description}
%         \SUBHEAD{if needed} ...
%         \SUBHEAD{if needed} ...
%          ...
%       \HANDIN{What to hand in and how}
%       \begin{checklist}
%       \item ...
%       \end{checklist}
% There is no need to include a "\documentstyle."
% However, there should be an "\end{document}."
%
%===========================================================
\documentclass[11pt,twoside,titlepage]{article}
%%NEED TO ADD epsf!!
\usepackage{threeparttop}
\usepackage{graphicx}
\usepackage{latexsym}
\usepackage{color}
\usepackage{listings}
\usepackage{fancyvrb}
%\usepackage{pgf,pgfarrows,pgfnodes,pgfautomata,pgfheaps,pgfshade}
\usepackage{tikz}
\usepackage[normalem]{ulem}
\tikzset{
    %Define standard arrow tip
%    >=stealth',
    %Define style for boxes
    oval/.style={
           rectangle,
           rounded corners,
           draw=black, very thick,
           text width=6.5em,
           minimum height=2em,
           text centered},
    % Define arrow style
    arr/.style={
           ->,
           thick,
           shorten <=2pt,
           shorten >=2pt,}
}
\usepackage[noend]{algorithmic}
\usepackage[noend]{algorithm}
\newcommand{\bfor}{{\bf for\ }}
\newcommand{\bthen}{{\bf then\ }}
\newcommand{\bwhile}{{\bf while\ }}
\newcommand{\btrue}{{\bf true\ }}
\newcommand{\bfalse}{{\bf false\ }}
\newcommand{\bto}{{\bf to\ }}
\newcommand{\bdo}{{\bf do\ }}
\newcommand{\bif}{{\bf if\ }}
\newcommand{\belse}{{\bf else\ }}
\newcommand{\band}{{\bf and\ }}
\newcommand{\breturn}{{\bf return\ }}
\newcommand{\mod}{{\rm mod}}
\renewcommand{\algorithmiccomment}[1]{$\rhd$ #1}
\newenvironment{checklist}{\par\noindent\hspace{-.25in}{\bf Checklist:}\renewcommand{\labelitemi}{$\Box$}%
\begin{itemize}}{\end{itemize}}
\pagestyle{threepartheadings}
\usepackage{url}
\usepackage{wrapfig}
% \usepackage{hyperref}
\usepackage[hidelinks]{hyperref}
%=========================
% One-inch margins everywhere
%=========================
\setlength{\topmargin}{0in}
\setlength{\textheight}{8.5in}
\setlength{\oddsidemargin}{0in}
\setlength{\evensidemargin}{0in}
\setlength{\textwidth}{6.5in}
%===============================
%===============================
% Macro for document title:
%===============================
\newcommand{\MYTITLE}[1]%
   {\begin{center}
     \begin{center}
     \bf
     FS 102 \\Software Everywhere\\
     Spring 2017\\
     \medskip
     \end{center}
     \bf
     #1
     \end{center}
}
%================================
% Macro for headings:
%================================
\newcommand{\MYHEADERS}[2]%
   {\lhead{#1}
    \rhead{#2}
    \immediate\write16{}
    \immediate\write16{DATE OF HANDOUT?}
    \read16 to \dateofhandout
    \lfoot{\sc Handed out on \dateofhandout}
    \immediate\write16{}
    \immediate\write16{HANDOUT NUMBER?}
    \read16 to\handoutnum
    \rfoot{Handout \handoutnum}
   }

%================================
% Macro for bold italic:
%================================
\newcommand{\bit}[1]{{\textit{\textbf{#1}}}}

%=========================
% Non-zero paragraph skips.
%=========================
\setlength{\parskip}{1ex}

%=========================
% Create various environments:
%=========================
\newcommand{\PURPOSE}{\par\noindent\hspace{-.25in}{\bf Purpose:\ }}
\newcommand{\SUMMARY}{\par\noindent\hspace{-.25in}{\bf Summary:\ }}
\newcommand{\DETAILS}{\par\noindent\hspace{-.25in}{\bf Details:\ }}
\newcommand{\HANDIN}{\par\noindent\hspace{-.25in}{\bf Hand in:\ }}
\newcommand{\SUBHEAD}[1]{\bigskip\par\noindent\hspace{-.1in}{\sc #1}\\}
%\newenvironment{CHECKLIST}{\begin{itemize}}{\end{itemize}}


\usepackage[compact]{titlesec}

\begin{document}

\MYTITLE{Assignment 2\\Assigned: Wednesday, February 8, 2017\\Due: Wednesday, February 15, 2017 at the start of class}

\vspace*{-.2in}
\section*{Introduction}

Now that you have created the first version of your mobile-ready web site and the site is ``live'', you are ready to
begin adding content to your site. In this assignment, you will write the first article that will be featured on your
web site. Consisting of between 600 and 1000 words, your article should contain four or five paragraphs of writing.
Additionally, your article will have both a header image that is released through Flickr under a Creative Commons
License and a license statement for the image. Along with having a captivating title, a TL;DR, and links to other
articles and web sites, your first featured article will be promoted through a social media campaign. In order to
successfully complete this assignment, please carefully adhere to the following guidelines.

\vspace*{-.1in}
\begin{itemize}
  \setlength{\itemsep}{-.01in}

\item {\bf If possible, use the laboratory computers.} If it is absolutely necessary for you to work on a different
  machine, be sure to regularly transfer your programs to the Alden machines and check their correctness. Please
  remember that, as stated in the syllabus, students should try to complete assignments using the specialized
  workstations in the laboratory. If you cannot use a laboratory computer, then please carefully explain the setup of
  your laptop to a teaching assistant or the course instructor when you are asking questions.

\item {\bf Follow each step carefully.} Slowly read each sentence in every assignment sheet, making sure that you
  precisely follow each instruction. Take notes about each step that you attempt, recording your questions and ideas
  and the challenges that you faced. If you are stuck, then please tell a teaching assistant or instructor what step
  you recently completed.

\item {\bf Regularly ask and answer questions.} Please log into Slack at the start of a class or practical session and
  then join the appropriate channel. If you have a question about one of the steps in an assignment, then you can post
  it to the designated channel. Or, you can ask a student sitting next to you or talk with a teaching assistant or the
  course instructor.

\item {\bf Store your files in Git}. Starting with this laboratory assignment, you will be responsible for storing all
  of your files in a Git repository. Please verify that you have saved your source code in your Git repository by
  typing ``{\tt git status}'' and ensuring everything is up to date.

\item {\bf Keep all of your files!} Don't delete your programs, output files, and reports after you hand them in---you
  will need them again later when you study for the quizzes and examinations and work on the other laboratory,
  practical, and final project assignments.

\item {\bf Back up your files regularly}. Use a flash drive, Google Drive, or your favorite backup method to keep a
  copy of your files in reserve. In the event of a system failure, you are responsible for ensuring that you have
  access to a recent backup copy of all your files.

\end{itemize}

\section*{Disciplined Writing with Version Control}

During this practical assignment and subsequent assignments, we will securely communicate with the GitHub.com servers
that will host all of our projects. If you have not done so already, please perform all of the steps to configure the
accounts on the departmental servers and the GitHub service. Throughout this assignment, you should refer to the
following web site for additional information: \url{https://guides.github.com/activities/hello-world/}. As you will be
required to use Git in all of the writing, speaking, and practical assignments and during the class sessions, please be
sure to keep a record of all of the steps that you complete and the challenges that you face. You should also
communicate, under the confines of the Honor Code, with other students to ensure that everyone is able to
successfully complete each of the steps outlined in this assignment.

\begin{enumerate}

  \itemsep 0em

  \item If you do not already have a GitHub account, then please go to the GitHub web site and create one---make
    sure that you use your {\tt allegheny.edu} email address so that you can join the GitHub Educational Community as
    this step becomes necessary. Also, please make sure that you add a description of yourself and an appropriate
    professional photograph to your GitHub profile. For examples of what a professional GitHub profile might look like,
    please consider studying {\tt https://github.com/una} and {\tt https://github.com/gkapfham}.

  \item If you have not already done so, please make sure that your SSH key is correctly uploaded to the GitHub servers
    and that you are ready to create your own GitHub repository. If you are not sure how to complete this step, please
    ask the course instructor or a teaching assistant.

  \item Again, when you are completing these steps, please make sure that you take careful notes about the inputs,
    outputs, and behavior of each command. If there is something that you do not understand, then please ask the course
    instructor or the teaching assistant about it.

  \item Since this is your first writing assignment and you are still learning how to use the appropriate software,
    don't become frustrated if you make a mistake. Instead, use your mistakes as an opportunity for learning both about
    the necessary technology and the background and expertise of the other students in the class, the teaching
    assistants, and the course instructor. Remember, you can use Slack to talk with the instructor by using ``{\tt
    @gkapfham}'' in a channel.

  \item Now, use your web browser to go to your GitHub profile and click the ``Repositories'' label near the top of the
    screen. Next, find and click the green button with the ``New'' label. At this point, you can create a new repository
    with the name ``about-me'', ensuring that it is a public repository that is initialized with an empty {\tt
    README.md} file. You can now click the ``Edit'' button and type in a brief description for this repository that will
    contain an article that introduces yourself to your audience. In a subsequent assignment, we will add the content
    that you create for this assignment to your own mobile-ready web site.

  \item Now that your Git repository is initialized on the GitHub servers, it is time to ``clone'' it so that it is
    editable on laboratory computer. Click the green ``Clone'' button, highlight the command in the text area, copy this
    command to the clipboard, and open your terminal window. If you have not done so already, then create a {\tt
    fs102S017} directory and change into it. Next, type the words ``{\tt git clone}'' in your terminal window and then
    paste the command that you copied from GitHub's interface. At this point, you should see that your repository has
    downloaded to your computer from the GitHub server. Please see the course instructor or a teaching assistant if you
    are not able to successfully complete this step.

  \item Make sure that you can find the {\tt README.md} file that has been downloaded to your computer. To edit this
    file, you will need to use the {\tt atom} text editor by typing ``{\tt atom README.md}'' in your terminal window.
    After you modify this file --- perhaps by adding a new sentence or revising and existing sentence --- you should use
    the ``{\tt git add}'' and ``{\tt git commit}'' and ``{\tt git push}'' commands to transfer your content to the
    GitHub servers. Please make sure that you write and revise in an incremental fashion, committing and pushing your
    sentences on a regular basis --- generally, small modifications are bigger than large ones because they make it
    easier for you and your editor to track and comment on your developing article.

  \item The writing assignment that you must complete requires you to write three to five paragraphs about yourself and
    how your life connects to the theme of this course, ``Software Everywhere''. To get some inspiration on how you
    should write this type of content, please review the ``Who is Guy?'' and ``Long Bio'' that is available at {\tt
    http://guykawasaki.com/guy-kawasaki/}.

  \item You should complete this assignment in collaboration with two editors. You will be responsible for asking these
    two editors to read your writing and give you feedback through the GitHub issue tracker. An editor will provide
    feedback through the issue tracker by clicking the ``Commits'' tab in GitHub's main interface, finding the commit
    number that contains the issue that you have identified in the writing, copying this number to the clipboard, and
    then using it when you report the issue. A writer will then consider the feedback, respond to it, and, when
    appropriate, resolve the issue that the editor raised. Each writer should also review the writing of the other two
    team members. While it is acceptable to read and mark printed versions of the article, all discussion should be
    visible in the GitHub issue tracker.

  \item Even though we will work on this assignment during at least two class sessions, you should budget enough time to
    outline, draft, discuss, and frequently revise this writing assignment. Remember, your focus should be on publishing
    an article that is concise, accessible, and error-free. Also, the main text of this writing assignment should be
    accompanied by a ``TL;DR'' or a ``Too Long; Didn't Read'' that summarizes the key points about yourself. Please see
    the instructor if you confront difficulties when using GitHub or if you have additional questions.

\end{enumerate}

\vspace*{-.05in}
\section*{Summary of the Required Deliverables}

This writing assignment invites you to complete the following tasks. After completing the assignment, please make sure
that you submit a printed version of your article to the course instructor.

\vspace*{-.1in}
\begin{enumerate}
  \setlength{\itemsep}{0in}

  \item Create a professional GitHub profile that will host all of your writing and presentations.
  \item Upload your ssh key to the GitHub system, in support of completing the later assignments.
  \item Create a Git repository called {\tt about-me} and download it to your computer workstation.
  \item Write a short three to five paragraph article introducing yourself and the course theme.
  \item Review, through multiple GitHub issues and a full-featured discussion, your peers' writing.

\end{enumerate}
\vspace*{-.1in}

In adherence to the Honor Code, students should complete this assignment on an individual basis. While it is appropriate
for students in this class to have high-level conversations about the assignment, it is necessary to distinguish
carefully between the student who discusses the principles underlying a problem with others and the student who produces
assignments that are identical to, or merely variations on, someone else's work. Any deliverables that are nearly
identical to the work of others will be taken as evidence of violating Allegheny College's \mbox{Honor Code}.


\end{document}
