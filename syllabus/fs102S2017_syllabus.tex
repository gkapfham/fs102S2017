% CS 580 style
% Typical usage (all UPPERCASE items are optional):
%       \input 580pre
%       \begin{document}
%       \MYTITLE{Title of document, e.g., Lab 1\\Due ...}
%       \MYHEADERS{short title}{other running head, e.g., due date}
%       \PURPOSE{Description of purpose}
%       \SUMMARY{Very short overview of assignment}
%       \DETAILS{Detailed description}
%         \SUBHEAD{if needed} ...
%         \SUBHEAD{if needed} ...
%          ...
%       \HANDIN{What to hand in and how}
%       \begin{checklist}
%       \item ...
%       \end{checklist}
% There is no need to include a "\documentstyle."
% However, there should be an "\end{document}."
%
%===========================================================
\documentclass[11pt,twoside,titlepage]{article}
%%NEED TO ADD epsf!!
\usepackage{threeparttop}
\usepackage{graphicx}
\usepackage{latexsym}
\usepackage{color}
\usepackage{listings}
\usepackage{fancyvrb}
%\usepackage{pgf,pgfarrows,pgfnodes,pgfautomata,pgfheaps,pgfshade}
\usepackage{tikz}
\usepackage[normalem]{ulem}
\tikzset{
    %Define standard arrow tip
%    >=stealth',
    %Define style for boxes
    oval/.style={
           rectangle,
           rounded corners,
           draw=black, very thick,
           text width=6.5em,
           minimum height=2em,
           text centered},
    % Define arrow style
    arr/.style={
           ->,
           thick,
           shorten <=2pt,
           shorten >=2pt,}
}
\usepackage[noend]{algorithmic}
\usepackage[noend]{algorithm}
\newcommand{\bfor}{{\bf for\ }}
\newcommand{\bthen}{{\bf then\ }}
\newcommand{\bwhile}{{\bf while\ }}
\newcommand{\btrue}{{\bf true\ }}
\newcommand{\bfalse}{{\bf false\ }}
\newcommand{\bto}{{\bf to\ }}
\newcommand{\bdo}{{\bf do\ }}
\newcommand{\bif}{{\bf if\ }}
\newcommand{\belse}{{\bf else\ }}
\newcommand{\band}{{\bf and\ }}
\newcommand{\breturn}{{\bf return\ }}
\newcommand{\mod}{{\rm mod}}
\renewcommand{\algorithmiccomment}[1]{$\rhd$ #1}
\newenvironment{checklist}{\par\noindent\hspace{-.25in}{\bf Checklist:}\renewcommand{\labelitemi}{$\Box$}%
\begin{itemize}}{\end{itemize}}
\pagestyle{threepartheadings}
\usepackage{url}
\usepackage{wrapfig}
% removing the standard hyperref to avoid the horrible boxes
%\usepackage{hyperref}
\usepackage[hidelinks]{hyperref}
% added in the dtklogos for the bibtex formatting
\usepackage{dtklogos}
%=========================
% One-inch margins everywhere
%=========================
\setlength{\topmargin}{0in}
\setlength{\textheight}{8.5in}
\setlength{\oddsidemargin}{0in}
\setlength{\evensidemargin}{0in}
\setlength{\textwidth}{6.5in}
%===============================
%===============================
% Macro for document title:
%===============================
\newcommand{\MYTITLE}[1]%
   {\begin{center}
     \begin{center}
     \bf
     FS 101\\Software Everywhere\\
     Fall 2013
     \medskip
     \end{center}
     \bf
     #1
     \end{center}
}
%================================
% Macro for headings:
%================================
\newcommand{\MYHEADERS}[2]%
   {\lhead{#1}
    \rhead{#2}
    %\immediate\write16{}
    %\immediate\write16{DATE OF HANDOUT?}
    %\read16 to \dateofhandout
    \def \dateofhandout {August 28, 2013}
    \lfoot{\sc Handed out on \dateofhandout}
    %\immediate\write16{}
    %\immediate\write16{HANDOUT NUMBER?}
    %\read16 to\handoutnum
    \def \handoutnum {1}
    \rfoot{Handout \handoutnum}
   }

%================================
% Macro for bold italic:
%================================
\newcommand{\bit}[1]{{\textit{\textbf{#1}}}}

%=========================
% Non-zero paragraph skips.
%=========================
\setlength{\parskip}{1ex}

%=========================
% Create various environments:
%=========================
\newcommand{\PURPOSE}{\par\noindent\hspace{-.25in}{\bf Purpose:\ }}
\newcommand{\SUMMARY}{\par\noindent\hspace{-.25in}{\bf Summary:\ }}
\newcommand{\DETAILS}{\par\noindent\hspace{-.25in}{\bf Details:\ }}
\newcommand{\HANDIN}{\par\noindent\hspace{-.25in}{\bf Hand in:\ }}
\newcommand{\SUBHEAD}[1]{\bigskip\par\noindent\hspace{-.1in}{\sc #1}\\}
%\newenvironment{CHECKLIST}{\begin{itemize}}{\end{itemize}}


\usepackage[compact]{titlesec}

\begin{document}
\MYTITLE{Syllabus}
\MYHEADERS{Syllabus}{}

\subsection*{Course Instructor}
Dr.\ Gregory M.\ Kapfhammer\\
\noindent Office Location: Alden Hall 108 \\
\noindent Office Phone: +1 814-332-2880 \\
\noindent Email: \url{gkapfham@allegheny.edu} \\
\noindent Twitter: \url{@GregKapfhammer} \\
\noindent Web Site: \url{http://www.cs.allegheny.edu/sites/gkapfham/}

\subsection*{Instructor's Office Hours}

\begin{itemize}
  \itemsep0em

  \item Monday, Wednesday, and Friday: 11:00 am--12:00 noon (15 minute time slots)

  \item Tuesday and Thursday: 10:00 am--11:00 am (15 minute time slots)

  \item Tuesday: 2:30 pm--3:30 pm (15 minute time slots)

  \item Friday: 4:00 pm--5:00 pm (15 minute time slots)

\end{itemize}

\vspace*{-.1in}

\noindent To schedule a meeting with me during my office hours, please visit my web site and click the ``Schedule'' link
in the top right-hand corner. Now, you can browse my office hours or schedule an appointment by clicking the correct
link and then reserving an open time slot. Students are also encouraged to post appropriate questions to a channel in
Slack, which is available at \url{https://FS111Spring2017.slack.com/}, and monitored by the instructor and the
teaching assistants.

\subsection*{Course Meeting Schedule}

Discussion, Presentation, and Group Work Session: Monday and Wednesday 3:00 pm--3:50 pm \\
Practical Session: Friday 3:00 pm--3:50 pm \\
Final Project Due Date: Tuesday, May 9, 2017 at 5:00 pm

\subsection*{Course Catalogue Description}

\begin{quote}

  An examination of the pervasive nature of computer software and the impact that it has on individuals and society.
  Participating in team-based and hands-on explorations of software systems, students examine the technical and ethical
  challenges facing a culture that relies heavily on computer software. Using state-of-the-art software technology to
  investigate topics such as social networking, online search, and digital music, students further develop their oral and
  written communication skills.

\end{quote}

\subsection*{Course Objectives}

Beyond the purpose of integrating new students into the intellectual life of Allegheny College, the goal of this course
is to ensure that students are able to effectively write and speak. Using the theme of ``Software Everywhere'' as an
ends towards achieving these goals, students will learn to write and speak in a variety of styles and on a multitude of
subjects. Students will also find out about the resources available to and rules and regulations upheld by students at
Allegheny College.

An additional objective of the course is to teach students how to use software to write and present about software
systems and the way in which they influence individual and society. Students will learn how to use the Git version
control system to manage their writing. A further goal for the course is to allow you to feature a portfolio of your
writing assignments and presentations on a mobile-ready web site hosted by GitHub. Finally, students will learn how to
use simple programming languages like Markdown and HTML to create their writing assignments and presentations.

\subsection*{Performance Objectives}

At the end of this course, students will better know how to manage their time in a fashion that will ensure their
academic success at Allegheny College. Members of the class should also be able to confidently write and revise
persuasive articles and give compelling presentations of no more than five minutes. The primary focus for the writing
assignments will be the production of text that is concise, accessible, and error-free. Unless specified otherwise, each
writing assignment will be accompanied by a ``TL;DR'' or a ``Too Long; Didn't Read'' that summarizes the key points of
each article. The primary focus of the presentations will be the delivery of short and engaging presentations that use
simple and appealing slides to convey a point. In addition to improving their writing and speaking skills, students
should know how to both participate in and lead class discussions, respectfully debate with their peers, actively
participate in group work, and interact with the course instructor through both digital mediums and face-to-face
interactions.

\subsection*{Required Textbooks}

% A Writer's Reference [Plastic Comb]
% Diana Hacker (Author), Nancy Sommers (Author)
% ISBN-10: 0312601433
% ISBN-13: 978-0312601430
% Seventh Edition
% Status: Required
% 18 copies

\noindent{\em A Writer's Reference}. Nancy Sommers. Seventh Edition, ISBN-10: 0312601433, ISBN-13: 978-0312601430,
245 pages, 2010. (From your First-Year/Sophomore 101 course).

% They Say, I Say: The Moves That Matter in Academic Writing [Paperback]
% Gerald Graff (Author), Cathy Birkenstein (Author)
% ISBN-10: 039393361X
% ISBN-13: 978-0393933611
% Second Edition
% Status: Required
% 18 copies

\noindent{\em They Say, I Say: The Moves That Matter in Academic Writing}. Gerald Graff and Cathy Birkenstein. Second Edition,
ISBN-10: 039393361X, ISBN-13: 978-0393933611, 245 pages, 2009.

% BUGS in Writing, Revised Edition: A Guide to Debugging Your Prose
% [Paperback]
% Lyn Dupre (Author)
% ISBN-10: 020137921X
% ISBN-13: 978-0201379211
% Second Edition
% Status: Required
% 18 copies

\noindent{\em BUGS in Writing: A Guide to Debugging Your Prose}. Lyn Dupr\'e. Second Edition, ISBN-10: 020137921X,
ISBN-13: 978-0201379211, 704 pages, 1998.

\noindent
Students wanting to hone their technical writing skills are encouraged to consult the following book.

\noindent{\em Writing for Computer Science}. Justin Zobel. Second Edition, ISBN-10: 1852338024, ISBN-13:
978-1852338022, 270 pages, 2004.

\noindent Along with reading the required books, you will be asked to study many additional articles from a wide variety
of conference proceedings, journals, web sites, and the popular press.

\subsection*{Class Policies}

\subsubsection*{Grading}

The grade that a student receives in this class will be based on the following categories. All percentages are
approximate and, if a need to do so presents itself, it is possible for the course instructor to change the assigned
percentages during the academic semester.

\begin{center}
  \begin{tabular}{ll}

    Class Participation and Instructor Meetings & 10\% \\
    Practical Assignments                       & 20\% \\
    Writing Assignments                         & 30\% \\
    Presentation Assignments                    & 15\% \\
    Final Project Portfolio                     & 10\%

  \end{tabular}
\end{center}

\noindent
\vspace*{-.1in}
These grading categories have the following definitions:

% \vspace*{-.05in}
\begin{itemize}

  \itemsep0em

  \item {\em Class Participation and Instructor Meetings\/}: All students are required to actively participate during
    all of the class sessions. Your participation will take forms such as answering questions about the required reading
    assignments, asking constructive questions of group members, giving presentations, and leading a discussion session.
    Furthermore, all students are required to meet with the course instructor during office hours for at least fifteen
    minutes during the Spring 2017 semester. These meetings must be scheduled through the instructor's reservation
    system and documented on a meeting record that you submit along with your final project portfolio. You also must
    regularly participate in the discussions on the Slack channels for this course. A student may request feedback on
    and will receive a final grade for this category.

  \item {\em Writing Assignments}: Students will write a wide variety of articles that they will feature on their
    mobile-ready web site. In addition to posting all assignments on your web site, students are responsible for
    submitting a printed and signed version of it at the start of the class session on which the assignment is due.
    Prior to the submission deadline for each assignment, there must be evidence in the student's GitHub repository ---
    in the form of GitHub issues and commits --- that document the discussion and revision of each article. Along with a
    grade for each assignment, the instructor will return a marked version of the article and, when appropriate, raise
    additional issues in the student's GitHub repository.

  \item {\em Presentation Assignments}: Students will give several presentations whose slides they will feature on their
    mobile-ready web site. In addition to posting all presentations on your web site, students are responsible for
    submitting an outline of it at the start of the class session on which the assignment is due. Prior to the date on
    which you will give your presentation, there must be evidence in your GitHub repository --- in the form of GitHub
    issues and commits --- that document the discussion and revision of each presentation. Along with a grade for each
    assignment, the instructor will return a marked version of the presentation outline and, when appropriate, raise
    additional issues in the student's GitHub repository.

  \item {\em Final Project Portfolio}: Leveraging the feedback that they received for each of their writing and
    presentation assignments, students will prepare a final project portfolio, available as a mobile-ready web site,
    that showcases the revised version of each assignment. In addition, you will prepare a final article that draws on
    all of your past writing and presentation assignments while still persuasively articulating your view of a
    course-related issue.

\end{itemize}
\vspace*{-.1in}

\subsubsection*{Assignment Submission}

All assignments will have a given due date. The printed version of the assignment is to be turned in at the beginning
of the class on that due date. Late assignments will be accepted for up to one week past the assigned due date with a
15\% penalty. All late assignments must be submitted at the beginning of the class that is scheduled one week after the
given due date. Unless special arrangements are made with the instructor, no assignments will be accepted after the
late deadline.

\subsubsection*{Attendance}

It is mandatory for all students to attend class.  If you will not be able to attend a class session, then please see
the instructor at least one week in advance to describe your situation.  Students who miss more than five unexcused
classes will have their final grade in the course reduced by one letter grade.  Students who miss more than ten
unexcused classes will automatically fail the course.

\vspace*{-.1in}
\subsubsection*{Class Preparation}

In order to minimize confusion and maximize learning, students must invest time to prepare for class discussions and
lectures.  During the class periods, the course instructor will often pose demanding questions that could require group
discussion, the creation of a program or test suite, a vote on a thought-provoking issue, or a group presentation.
Only students who have prepared for class by reading the assigned material and reviewing the current assignments will be
able to effectively participate in these discussions.  More importantly, only prepared students will be able to acquire
the knowledge and skills that are needed to be successful in both this course and subsequent courses.  In order to help
students remain organized and effectively prepare for classes, the course instructor will maintain a class schedule with
reading assignments and presentation slides.   During the class sessions students will also be required to download and
use programs, search the Web, modify the course Web site, and complete writing assignments.  Students who are
not comfortable with using Web browsers, document editors, and presentation programs should see the course instructor.

\subsubsection*{Email}

Using your Allegheny College email address, I will sometimes send out class announcements about matters such as
assignment clarifications or changes in the schedule. It is your responsibility to check your email at least once a day
and to ensure that you can reliably send and receive emails.

\subsubsection*{Disability Services}

The Americans with Disabilities Act (ADA) is a federal anti-discrimination statute that provides comprehensive civil
rights protection for persons with disabilities.  Among other things, this legislation requires all students with
disabilities be guaranteed a learning environment that provides for reasonable accommodation of their disabilities.
Students with disabilities who believe they may need accommodations in this class are encouraged to contact Disability
Services at 332-2898.  Disability Services is part of the Learning Commons and is located in Pelletier Library.
Please do this as soon as possible to ensure that approved accommodations are implemented in a timely fashion.

\subsubsection*{Honor Code}

The Academic Honor Program that governs the entire academic program at Allegheny College is described in the Allegheny
Course Catalogue.  The Honor Program applies to all work that is submitted for academic credit or to meet non-credit
requirements for graduation at Allegheny College.  This includes all work assigned for this class (e.g., examinations,
writing assignments, and presentations).  All students who have enrolled in the College will work under the Honor
Program.  Each student who has matriculated at the College has acknowledged the following pledge:

\vspace*{-.125in}
\begin{quote}
  I hereby recognize and pledge to fulfill my responsibilities, as defined in the Honor Code, and to maintain the
  integrity of both myself and the College community as a whole.
\end{quote}
\vspace*{-.175in}

\subsection*{Welcome to an Adventure in Software, Writing, and Speaking}

In reference to software, Frederick P.\ Brooks, Jr.\ wrote in {\em The Mythical Man Month}, ``The magic of myth and legend has come
true in our time.'' Since software is so pervasive and influential in our society and writing and speaking are so important,
I invite you to pursue with enthusiasm and vigor this adventure in software, writing, speaking, and the Allegheny
College community.



% Software is a pervasive aspect of our society that has the potential to positively influence people.


% Software is a pervasive aspect of our society that changes how we think and act.  High quality
% software also has the potential to positively influence the lives of people.
%
\end{document}
