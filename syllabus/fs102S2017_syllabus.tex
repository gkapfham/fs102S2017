% CS 580 style
% Typical usage (all UPPERCASE items are optional):
%       \input 580pre
%       \begin{document}
%       \MYTITLE{Title of document, e.g., Lab 1\\Due ...}
%       \MYHEADERS{short title}{other running head, e.g., due date}
%       \PURPOSE{Description of purpose}
%       \SUMMARY{Very short overview of assignment}
%       \DETAILS{Detailed description}
%         \SUBHEAD{if needed} ...
%         \SUBHEAD{if needed} ...
%          ...
%       \HANDIN{What to hand in and how}
%       \begin{checklist}
%       \item ...
%       \end{checklist}
% There is no need to include a "\documentstyle."
% However, there should be an "\end{document}."
%
%===========================================================
\documentclass[11pt,twoside,titlepage]{article}
%%NEED TO ADD epsf!!
\usepackage{threeparttop}
\usepackage{graphicx}
\usepackage{latexsym}
\usepackage{color}
\usepackage{listings}
\usepackage{fancyvrb}
%\usepackage{pgf,pgfarrows,pgfnodes,pgfautomata,pgfheaps,pgfshade}
\usepackage{tikz}
\usepackage[normalem]{ulem}
\tikzset{
    %Define standard arrow tip
%    >=stealth',
    %Define style for boxes
    oval/.style={
           rectangle,
           rounded corners,
           draw=black, very thick,
           text width=6.5em,
           minimum height=2em,
           text centered},
    % Define arrow style
    arr/.style={
           ->,
           thick,
           shorten <=2pt,
           shorten >=2pt,}
}
\usepackage[noend]{algorithmic}
\usepackage[noend]{algorithm}
\newcommand{\bfor}{{\bf for\ }}
\newcommand{\bthen}{{\bf then\ }}
\newcommand{\bwhile}{{\bf while\ }}
\newcommand{\btrue}{{\bf true\ }}
\newcommand{\bfalse}{{\bf false\ }}
\newcommand{\bto}{{\bf to\ }}
\newcommand{\bdo}{{\bf do\ }}
\newcommand{\bif}{{\bf if\ }}
\newcommand{\belse}{{\bf else\ }}
\newcommand{\band}{{\bf and\ }}
\newcommand{\breturn}{{\bf return\ }}
\newcommand{\mod}{{\rm mod}}
\renewcommand{\algorithmiccomment}[1]{$\rhd$ #1}
\newenvironment{checklist}{\par\noindent\hspace{-.25in}{\bf Checklist:}\renewcommand{\labelitemi}{$\Box$}%
\begin{itemize}}{\end{itemize}}
\pagestyle{threepartheadings}
\usepackage{url}
\usepackage{wrapfig}
% removing the standard hyperref to avoid the horrible boxes
%\usepackage{hyperref}
\usepackage[hidelinks]{hyperref}
% added in the dtklogos for the bibtex formatting
\usepackage{dtklogos}
%=========================
% One-inch margins everywhere
%=========================
\setlength{\topmargin}{0in}
\setlength{\textheight}{8.5in}
\setlength{\oddsidemargin}{0in}
\setlength{\evensidemargin}{0in}
\setlength{\textwidth}{6.5in}
%===============================
%===============================
% Macro for document title:
%===============================
\newcommand{\MYTITLE}[1]%
   {\begin{center}
     \begin{center}
     \bf
     FS 101\\Software Everywhere\\
     Fall 2013
     \medskip
     \end{center}
     \bf
     #1
     \end{center}
}
%================================
% Macro for headings:
%================================
\newcommand{\MYHEADERS}[2]%
   {\lhead{#1}
    \rhead{#2}
    %\immediate\write16{}
    %\immediate\write16{DATE OF HANDOUT?}
    %\read16 to \dateofhandout
    \def \dateofhandout {August 28, 2013}
    \lfoot{\sc Handed out on \dateofhandout}
    %\immediate\write16{}
    %\immediate\write16{HANDOUT NUMBER?}
    %\read16 to\handoutnum
    \def \handoutnum {1}
    \rfoot{Handout \handoutnum}
   }

%================================
% Macro for bold italic:
%================================
\newcommand{\bit}[1]{{\textit{\textbf{#1}}}}

%=========================
% Non-zero paragraph skips.
%=========================
\setlength{\parskip}{1ex}

%=========================
% Create various environments:
%=========================
\newcommand{\PURPOSE}{\par\noindent\hspace{-.25in}{\bf Purpose:\ }}
\newcommand{\SUMMARY}{\par\noindent\hspace{-.25in}{\bf Summary:\ }}
\newcommand{\DETAILS}{\par\noindent\hspace{-.25in}{\bf Details:\ }}
\newcommand{\HANDIN}{\par\noindent\hspace{-.25in}{\bf Hand in:\ }}
\newcommand{\SUBHEAD}[1]{\bigskip\par\noindent\hspace{-.1in}{\sc #1}\\}
%\newenvironment{CHECKLIST}{\begin{itemize}}{\end{itemize}}


\usepackage[compact]{titlesec}

\begin{document}
\MYTITLE{Syllabus}
\MYHEADERS{Syllabus}{}

\subsection*{Course Instructor}
Dr.\ Gregory M.\ Kapfhammer\\
\noindent Office Location: Alden Hall 108 \\
\noindent Office Phone: +1 814-332-2880 \\
\noindent Email: \url{gkapfham@allegheny.edu} \\
\noindent Twitter: \url{@GregKapfhammer} \\
\noindent Web Site: \url{http://www.cs.allegheny.edu/sites/gkapfham/}

\subsection*{Instructor's Office Hours}

\begin{itemize}
  \itemsep0em

  \item Monday, Wednesday, and Friday: 11:00 am--12:00 noon (15 minute time slots)

  \item Tuesday and Thursday: 10:00 am--11:00 am (15 minute time slots)

  \item Tuesday: 2:30 pm--3:30 pm (15 minute time slots)

  \item Friday: 4:00 pm--5:00 pm (15 minute time slots)

\end{itemize}

\vspace*{-.1in}

\noindent To schedule a meeting with me during my office hours, please visit my web site and click the ``Schedule'' link
in the top right-hand corner. Now, you can browse my office hours or schedule an appointment by clicking the correct
link and then reserving an open time slot. Students are also encouraged to post appropriate questions to a channel in
Slack, which is available at \url{https://FS111Spring2017.slack.com/}, and monitored by the instructor and the
teaching assistants.

\subsection*{Course Meeting Schedule}

Discussion, Presentation, and Group Work Session: Monday and Wednesday 3:00 pm--3:50 pm \\
Practical Session: Friday 3:00 pm--3:50 pm \\
Final Project Due Date: Tuesday, May 9, 2017 at 5:00 pm

\subsection*{Course Catalogue Description}

\begin{quote}

An examination of the pervasive nature of computer software and the impact that computer technology has on society.
Drawing on articles from the popular press and the computer science literature, this course examines the technical and
ethical challenges that face a culture that regularly uses computer software applications. Sample topics include the
Internet, Google, online music, open source software, electronic commerce, social networking, and data mining.
Coursework emphasizes the development of effective oral and written communication skills with a focus on description,
summary, and critical thinking.

\end{quote}

\subsection*{Course Objectives}

Beyond the purpose of integrating new students into the intellectual life of Allegheny College, the goal of this course
is to ensure that students are able to effectively write and speak.  Using the theme of ``Software Everywhere'' as an
ends towards achieving these goals, students will learn to write and speak in a variety of styles and on a multitude of
subjects.  Students will also find out about the resources available to and rules and regulations upheld by students at
Allegheny College.  Finally, students will learn how to competently interact with both their adviser and other
professors.

\subsection*{Performance Objectives}

At the end of this course, students will know how to manage their time in a fashion that will ensure their academic
success at Allegheny College.  Members of the class should be able to confidently write and revise high-quality papers
of less than ten pages and practice and give interesting, exciting, and accurate presentations of no more than five
minutes.  Students should know how to both participate in and lead class discussions, respectfully debate their peers,
actively participate in group work, and interact with the course instructor during both class and advising sessions.

\subsection*{Required Textbooks}

  % A Writer's Reference [Plastic Comb]
  % Diana Hacker (Author), Nancy Sommers (Author)
  % ISBN-10: 0312601433
  % ISBN-13: 978-0312601430
  % Seventh Edition
  % Status: Required
  % 18 copies

  \noindent{\em A Writer's Reference}. Nancy Sommers. Seventh Edition,  ISBN-10: 0312601433, ISBN-13: 978-0312601430,
  245 pages, 2010.

  % They Say, I Say: The Moves That Matter in Academic Writing [Paperback]
  % Gerald Graff (Author), Cathy Birkenstein (Author)
  % ISBN-10: 039393361X
  % ISBN-13: 978-0393933611
  % Second Edition
  % Status: Required
  % 18 copies

  \noindent{\em They Say, I Say: The Moves That Matter in Academic Writing}. Gerald Graff and Cathy Birkenstein. Second Edition,
  ISBN-10: 039393361X, ISBN-13: 978-0393933611, 245 pages, 2009.

  % BUGS in Writing, Revised Edition: A Guide to Debugging Your Prose
  % [Paperback]
  % Lyn Dupre (Author)
  % ISBN-10: 020137921X
  % ISBN-13: 978-0201379211
  % Second Edition
  % Status: Required
  % 18 copies

\noindent{\em BUGS in Writing: A Guide to Debugging Your Prose}. Lyn Dupr\'e. Second Edition,  ISBN-10: 020137921X,
ISBN-13: 978-0201379211, 704 pages, 1998.

\noindent
Students wanting to hone their technical writing skills are encouraged to consult the following book.

\noindent{\em Writing for Computer Science}.  Justin Zobel. Second Edition,  ISBN-10: 1852338024, ISBN-13:
978-1852338022, 270 pages, 2004.

\noindent
Along with reading the required books, you will be asked to study many additional articles from a wide variety of
conference proceedings, journals, and the popular press.

\subsection*{Class Policies}

\subsubsection*{Grading}

The grade that a student receives in this class will be based on the following categories. All percentages are
approximate and, if a need to do so presents itself, it is possible for the assigned percentages to change during the
academic semester.

\begin{center}
\begin{tabular}{ll}
Class Participation & 10\% \\
Instructor Meetings & 10\% \\
Short Writing Assignments & 40\% \\
Long Writing Assignments & 15\% \\
Presentation Assignments & 15\% \\
Final Examination & 10\%
\end{tabular}
\end{center}

\noindent
\vspace*{-.1in}
These grading categories have the following definitions:

\vspace*{-.05in}
\begin{itemize}

    \itemsep0em
    \item {\em Class Participation}: All students are required to actively participate during
        all of the class sessions. Your participation will take forms such as answering questions about the required
        reading assignments, responding to comments made by members of the class, asking constructive questions of your
        group members, giving presentations, and leading a discussion session. A student will receive an interim and
        final grade for this category.

    \item {\em Instructor Meetings}: All students are required to meet with the course instructor during office
        hours for a total of sixty minutes during the Fall 2013 semester.  These meetings must be scheduled through the
        course instructor's reservation system and documented on a meeting record that you submit on the day of the final
        examination.  Thirty minutes of these meetings must be devoted to an advising session during which the student
        will plan their schedule for the upcoming Spring 2014 academic semester.  Students must schedule their advising
        meeting no later than Friday, September 13, 2013; after picking a time for this meeting students should submit
        printed evidence of its existence to the course instructor.

    \item {\em Short and Long Writing Assignments}: Throughout this class, students will write a wide variety of short
        and long writing assignments.  While all assignments will be posted on the course Web site, students are
        responsible for submitting a printed and signed version of their assignment at the start of the class session on
        which the assignment is due.  To ensure that they can turn in a high quality paper, students must start writing
        and revising these assignments and seeking feedback from the course instructor well in advance of the submission
        deadline.

    \item {\em Presentation Assignments}: Students will give several in-class presentations during this semester. While
        the assignment sheet will state the days on which the presentations will be given, the order in which students
        give their presentations will be randomly generated and revealed incrementally.  Unless there are severe
        extenuating circumstances, a failing grade will be given to students who are not in class on the day that they
        are invited to give their presentation.

    \item {\em Final Examination}: The final examination is a three-hour cumulative test.  By enrolling in this course,
        students agree that, unless there are severe extenuating circumstances, they will take the final examination at
        the time stated on the first page of the syllabus.  The final examination ensures that students have a basic
        understanding of both the software-related topics and the writing and speaking skills that were the focus of the
        course.

\end{itemize}
\vspace*{-.2in}

\subsubsection*{Assignment Submission}

All assignments will have a given due date.  The printed version of the assignment is to be turned in at the beginning
of the class on that due date.  Late assignments will be accepted for up to one week past the assigned due date with a
10\% penalty.  All late assignments must be submitted at the beginning of the class that is scheduled one week after the
given due date.  Unless special arrangements are made with the instructor, no assignments will be accepted after the
late deadline.

\subsubsection*{Attendance}

It is mandatory for all students to attend class.  If you will not be able to attend a class session, then please see
the instructor at least one week in advance to describe your situation.  Students who miss more than five unexcused
classes will have their final grade in the course reduced by one letter grade.  Students who miss more than ten
unexcused classes will automatically fail the course.

\vspace*{-.2in}
\subsubsection*{Class Preparation}

% The study of the computer science discipline is very challenging.  Students in this class will be challenged to learn
% the principles and practice of software development.  During the coming semester even the most diligent student will
% experience times of frustration when they are attempting to understand a challenging concept or complete a difficult
% laboratory assignment.  In many situations some of the material that we examine will initially be confusing : do not
% despair!  Press on and persevere!
%

In order to minimize confusion and maximize learning, students must invest time to prepare for class discussions and
lectures.  During the class periods, the course instructor will often pose demanding questions that could require group
discussion, the creation of a program or test suite, a vote on a thought-provoking issue, or a group presentation.
Only students who have prepared for class by reading the assigned material and reviewing the current assignments will be
able to effectively participate in these discussions.  More importantly, only prepared students will be able to acquire
the knowledge and skills that are needed to be successful in both this course and subsequent courses.  In order to help
students remain organized and effectively prepare for classes, the course instructor will maintain a class schedule with
reading assignments and presentation slides.   During the class sessions students will also be required to download and
use programs, search the Web, modify the course Web site, and complete writing assignments.  Students who are
not comfortable with using Web browsers, document editors, and presentation programs should see the course instructor.

\subsubsection*{Email}

Using your Allegheny College email address, I will sometimes send out class announcements about matters such as
assignment clarifications or changes in the schedule. It is your responsibility to check your email at least once a day
and to ensure that you can reliably send and receive emails.

\subsubsection*{Disability Services}

The Americans with Disabilities Act (ADA) is a federal anti-discrimination statute that provides comprehensive civil
rights protection for persons with disabilities.  Among other things, this legislation requires all students with
disabilities be guaranteed a learning environment that provides for reasonable accommodation of their disabilities.
Students with disabilities who believe they may need accommodations in this class are encouraged to contact Disability
Services at 332-2898.  Disability Services is part of the Learning Commons and is located in Pelletier Library.
Please do this as soon as possible to ensure that approved accommodations are implemented in a timely fashion.

\subsubsection*{Honor Code}

The Academic Honor Program that governs the entire academic program at Allegheny College is described in the Allegheny
Course Catalogue.  The Honor Program applies to all work that is submitted for academic credit or to meet non-credit
requirements for graduation at Allegheny College.  This includes all work assigned for this class (e.g., examinations,
writing assignments, and presentations).  All students who have enrolled in the College will work under the Honor
Program.  Each student who has matriculated at the College has acknowledged the following pledge:

\vspace*{-.125in}
\begin{quote}
I hereby recognize and pledge to fulfill my responsibilities, as defined in the Honor Code, and to maintain the
integrity of both myself and the College community as a whole.
\end{quote}
\vspace*{-.175in}

\subsection*{Welcome to an Adventure in Software, Writing, and Speaking}

In reference to software, Frederick P.\ Brooks, Jr.\ wrote in {\em The Mythical Man Month}, ``The magic of myth and legend has come
true in our time.'' Since software is so pervasive and influential in our society and writing and speaking are so important,
I invite you to pursue with enthusiasm and vigor this adventure in software, writing, speaking, and the Allegheny
College community.



% Software is a pervasive aspect of our society that has the potential to positively influence people.


% Software is a pervasive aspect of our society that changes how we think and act.  High quality
% software also has the potential to positively influence the lives of people.
%
\end{document}
